\documentclass[11pt]{article}

    \usepackage[breakable]{tcolorbox}
    \usepackage{parskip} % Stop auto-indenting (to mimic markdown behaviour)
    

    % Basic figure setup, for now with no caption control since it's done
    % automatically by Pandoc (which extracts ![](path) syntax from Markdown).
    \usepackage{graphicx}
    % Maintain compatibility with old templates. Remove in nbconvert 6.0
    \let\Oldincludegraphics\includegraphics
    % Ensure that by default, figures have no caption (until we provide a
    % proper Figure object with a Caption API and a way to capture that
    % in the conversion process - todo).
    \usepackage{caption}
    \DeclareCaptionFormat{nocaption}{}
    \captionsetup{format=nocaption,aboveskip=0pt,belowskip=0pt}

    \usepackage{float}
    \floatplacement{figure}{H} % forces figures to be placed at the correct location
    \usepackage{xcolor} % Allow colors to be defined
    \usepackage{enumerate} % Needed for markdown enumerations to work
    \usepackage{geometry} % Used to adjust the document margins
    \usepackage{amsmath} % Equations
    \usepackage{amssymb} % Equations
    \usepackage{textcomp} % defines textquotesingle
    % Hack from http://tex.stackexchange.com/a/47451/13684:
    \AtBeginDocument{%
        \def\PYZsq{\textquotesingle}% Upright quotes in Pygmentized code
    }
    \usepackage{upquote} % Upright quotes for verbatim code
    \usepackage{eurosym} % defines \euro

    \usepackage{iftex}
    \ifPDFTeX
        \usepackage[T1]{fontenc}
        \IfFileExists{alphabeta.sty}{
              \usepackage{alphabeta}
          }{
              \usepackage[mathletters]{ucs}
              \usepackage[utf8x]{inputenc}
          }
    \else
        \usepackage{fontspec}
        \usepackage{unicode-math}
    \fi

    \usepackage{fancyvrb} % verbatim replacement that allows latex
    \usepackage{grffile} % extends the file name processing of package graphics
                         % to support a larger range
    \makeatletter % fix for old versions of grffile with XeLaTeX
    \@ifpackagelater{grffile}{2019/11/01}
    {
      % Do nothing on new versions
    }
    {
      \def\Gread@@xetex#1{%
        \IfFileExists{"\Gin@base".bb}%
        {\Gread@eps{\Gin@base.bb}}%
        {\Gread@@xetex@aux#1}%
      }
    }
    \makeatother
    \usepackage[Export]{adjustbox} % Used to constrain images to a maximum size
    \adjustboxset{max size={0.9\linewidth}{0.9\paperheight}}

    % The hyperref package gives us a pdf with properly built
    % internal navigation ('pdf bookmarks' for the table of contents,
    % internal cross-reference links, web links for URLs, etc.)
    \usepackage{hyperref}
    % The default LaTeX title has an obnoxious amount of whitespace. By default,
    % titling removes some of it. It also provides customization options.
    \usepackage{titling}
    \usepackage{longtable} % longtable support required by pandoc >1.10
    \usepackage{booktabs}  % table support for pandoc > 1.12.2
    \usepackage{array}     % table support for pandoc >= 2.11.3
    \usepackage{calc}      % table minipage width calculation for pandoc >= 2.11.1
    \usepackage[inline]{enumitem} % IRkernel/repr support (it uses the enumerate* environment)
    \usepackage[normalem]{ulem} % ulem is needed to support strikethroughs (\sout)
                                % normalem makes italics be italics, not underlines
    \usepackage{soul}      % strikethrough (\st) support for pandoc >= 3.0.0
    \usepackage{mathrsfs}
    

    
    % Colors for the hyperref package
    \definecolor{urlcolor}{rgb}{0,.145,.698}
    \definecolor{linkcolor}{rgb}{.71,0.21,0.01}
    \definecolor{citecolor}{rgb}{.12,.54,.11}

    % ANSI colors
    \definecolor{ansi-black}{HTML}{3E424D}
    \definecolor{ansi-black-intense}{HTML}{282C36}
    \definecolor{ansi-red}{HTML}{E75C58}
    \definecolor{ansi-red-intense}{HTML}{B22B31}
    \definecolor{ansi-green}{HTML}{00A250}
    \definecolor{ansi-green-intense}{HTML}{007427}
    \definecolor{ansi-yellow}{HTML}{DDB62B}
    \definecolor{ansi-yellow-intense}{HTML}{B27D12}
    \definecolor{ansi-blue}{HTML}{208FFB}
    \definecolor{ansi-blue-intense}{HTML}{0065CA}
    \definecolor{ansi-magenta}{HTML}{D160C4}
    \definecolor{ansi-magenta-intense}{HTML}{A03196}
    \definecolor{ansi-cyan}{HTML}{60C6C8}
    \definecolor{ansi-cyan-intense}{HTML}{258F8F}
    \definecolor{ansi-white}{HTML}{C5C1B4}
    \definecolor{ansi-white-intense}{HTML}{A1A6B2}
    \definecolor{ansi-default-inverse-fg}{HTML}{FFFFFF}
    \definecolor{ansi-default-inverse-bg}{HTML}{000000}

    % common color for the border for error outputs.
    \definecolor{outerrorbackground}{HTML}{FFDFDF}

    % commands and environments needed by pandoc snippets
    % extracted from the output of `pandoc -s`
    \providecommand{\tightlist}{%
      \setlength{\itemsep}{0pt}\setlength{\parskip}{0pt}}
    \DefineVerbatimEnvironment{Highlighting}{Verbatim}{commandchars=\\\{\}}
    % Add ',fontsize=\small' for more characters per line
    \newenvironment{Shaded}{}{}
    \newcommand{\KeywordTok}[1]{\textcolor[rgb]{0.00,0.44,0.13}{\textbf{{#1}}}}
    \newcommand{\DataTypeTok}[1]{\textcolor[rgb]{0.56,0.13,0.00}{{#1}}}
    \newcommand{\DecValTok}[1]{\textcolor[rgb]{0.25,0.63,0.44}{{#1}}}
    \newcommand{\BaseNTok}[1]{\textcolor[rgb]{0.25,0.63,0.44}{{#1}}}
    \newcommand{\FloatTok}[1]{\textcolor[rgb]{0.25,0.63,0.44}{{#1}}}
    \newcommand{\CharTok}[1]{\textcolor[rgb]{0.25,0.44,0.63}{{#1}}}
    \newcommand{\StringTok}[1]{\textcolor[rgb]{0.25,0.44,0.63}{{#1}}}
    \newcommand{\CommentTok}[1]{\textcolor[rgb]{0.38,0.63,0.69}{\textit{{#1}}}}
    \newcommand{\OtherTok}[1]{\textcolor[rgb]{0.00,0.44,0.13}{{#1}}}
    \newcommand{\AlertTok}[1]{\textcolor[rgb]{1.00,0.00,0.00}{\textbf{{#1}}}}
    \newcommand{\FunctionTok}[1]{\textcolor[rgb]{0.02,0.16,0.49}{{#1}}}
    \newcommand{\RegionMarkerTok}[1]{{#1}}
    \newcommand{\ErrorTok}[1]{\textcolor[rgb]{1.00,0.00,0.00}{\textbf{{#1}}}}
    \newcommand{\NormalTok}[1]{{#1}}

    % Additional commands for more recent versions of Pandoc
    \newcommand{\ConstantTok}[1]{\textcolor[rgb]{0.53,0.00,0.00}{{#1}}}
    \newcommand{\SpecialCharTok}[1]{\textcolor[rgb]{0.25,0.44,0.63}{{#1}}}
    \newcommand{\VerbatimStringTok}[1]{\textcolor[rgb]{0.25,0.44,0.63}{{#1}}}
    \newcommand{\SpecialStringTok}[1]{\textcolor[rgb]{0.73,0.40,0.53}{{#1}}}
    \newcommand{\ImportTok}[1]{{#1}}
    \newcommand{\DocumentationTok}[1]{\textcolor[rgb]{0.73,0.13,0.13}{\textit{{#1}}}}
    \newcommand{\AnnotationTok}[1]{\textcolor[rgb]{0.38,0.63,0.69}{\textbf{\textit{{#1}}}}}
    \newcommand{\CommentVarTok}[1]{\textcolor[rgb]{0.38,0.63,0.69}{\textbf{\textit{{#1}}}}}
    \newcommand{\VariableTok}[1]{\textcolor[rgb]{0.10,0.09,0.49}{{#1}}}
    \newcommand{\ControlFlowTok}[1]{\textcolor[rgb]{0.00,0.44,0.13}{\textbf{{#1}}}}
    \newcommand{\OperatorTok}[1]{\textcolor[rgb]{0.40,0.40,0.40}{{#1}}}
    \newcommand{\BuiltInTok}[1]{{#1}}
    \newcommand{\ExtensionTok}[1]{{#1}}
    \newcommand{\PreprocessorTok}[1]{\textcolor[rgb]{0.74,0.48,0.00}{{#1}}}
    \newcommand{\AttributeTok}[1]{\textcolor[rgb]{0.49,0.56,0.16}{{#1}}}
    \newcommand{\InformationTok}[1]{\textcolor[rgb]{0.38,0.63,0.69}{\textbf{\textit{{#1}}}}}
    \newcommand{\WarningTok}[1]{\textcolor[rgb]{0.38,0.63,0.69}{\textbf{\textit{{#1}}}}}


    % Define a nice break command that doesn't care if a line doesn't already
    % exist.
    \def\br{\hspace*{\fill} \\* }
    % Math Jax compatibility definitions
    \def\gt{>}
    \def\lt{<}
    \let\Oldtex\TeX
    \let\Oldlatex\LaTeX
    \renewcommand{\TeX}{\textrm{\Oldtex}}
    \renewcommand{\LaTeX}{\textrm{\Oldlatex}}
    % Document parameters
    % Document title
    \title{QuantumLinearAlgebra\_HHL\_Final}
    
    
    
    
    
    
    
% Pygments definitions
\makeatletter
\def\PY@reset{\let\PY@it=\relax \let\PY@bf=\relax%
    \let\PY@ul=\relax \let\PY@tc=\relax%
    \let\PY@bc=\relax \let\PY@ff=\relax}
\def\PY@tok#1{\csname PY@tok@#1\endcsname}
\def\PY@toks#1+{\ifx\relax#1\empty\else%
    \PY@tok{#1}\expandafter\PY@toks\fi}
\def\PY@do#1{\PY@bc{\PY@tc{\PY@ul{%
    \PY@it{\PY@bf{\PY@ff{#1}}}}}}}
\def\PY#1#2{\PY@reset\PY@toks#1+\relax+\PY@do{#2}}

\@namedef{PY@tok@w}{\def\PY@tc##1{\textcolor[rgb]{0.73,0.73,0.73}{##1}}}
\@namedef{PY@tok@c}{\let\PY@it=\textit\def\PY@tc##1{\textcolor[rgb]{0.24,0.48,0.48}{##1}}}
\@namedef{PY@tok@cp}{\def\PY@tc##1{\textcolor[rgb]{0.61,0.40,0.00}{##1}}}
\@namedef{PY@tok@k}{\let\PY@bf=\textbf\def\PY@tc##1{\textcolor[rgb]{0.00,0.50,0.00}{##1}}}
\@namedef{PY@tok@kp}{\def\PY@tc##1{\textcolor[rgb]{0.00,0.50,0.00}{##1}}}
\@namedef{PY@tok@kt}{\def\PY@tc##1{\textcolor[rgb]{0.69,0.00,0.25}{##1}}}
\@namedef{PY@tok@o}{\def\PY@tc##1{\textcolor[rgb]{0.40,0.40,0.40}{##1}}}
\@namedef{PY@tok@ow}{\let\PY@bf=\textbf\def\PY@tc##1{\textcolor[rgb]{0.67,0.13,1.00}{##1}}}
\@namedef{PY@tok@nb}{\def\PY@tc##1{\textcolor[rgb]{0.00,0.50,0.00}{##1}}}
\@namedef{PY@tok@nf}{\def\PY@tc##1{\textcolor[rgb]{0.00,0.00,1.00}{##1}}}
\@namedef{PY@tok@nc}{\let\PY@bf=\textbf\def\PY@tc##1{\textcolor[rgb]{0.00,0.00,1.00}{##1}}}
\@namedef{PY@tok@nn}{\let\PY@bf=\textbf\def\PY@tc##1{\textcolor[rgb]{0.00,0.00,1.00}{##1}}}
\@namedef{PY@tok@ne}{\let\PY@bf=\textbf\def\PY@tc##1{\textcolor[rgb]{0.80,0.25,0.22}{##1}}}
\@namedef{PY@tok@nv}{\def\PY@tc##1{\textcolor[rgb]{0.10,0.09,0.49}{##1}}}
\@namedef{PY@tok@no}{\def\PY@tc##1{\textcolor[rgb]{0.53,0.00,0.00}{##1}}}
\@namedef{PY@tok@nl}{\def\PY@tc##1{\textcolor[rgb]{0.46,0.46,0.00}{##1}}}
\@namedef{PY@tok@ni}{\let\PY@bf=\textbf\def\PY@tc##1{\textcolor[rgb]{0.44,0.44,0.44}{##1}}}
\@namedef{PY@tok@na}{\def\PY@tc##1{\textcolor[rgb]{0.41,0.47,0.13}{##1}}}
\@namedef{PY@tok@nt}{\let\PY@bf=\textbf\def\PY@tc##1{\textcolor[rgb]{0.00,0.50,0.00}{##1}}}
\@namedef{PY@tok@nd}{\def\PY@tc##1{\textcolor[rgb]{0.67,0.13,1.00}{##1}}}
\@namedef{PY@tok@s}{\def\PY@tc##1{\textcolor[rgb]{0.73,0.13,0.13}{##1}}}
\@namedef{PY@tok@sd}{\let\PY@it=\textit\def\PY@tc##1{\textcolor[rgb]{0.73,0.13,0.13}{##1}}}
\@namedef{PY@tok@si}{\let\PY@bf=\textbf\def\PY@tc##1{\textcolor[rgb]{0.64,0.35,0.47}{##1}}}
\@namedef{PY@tok@se}{\let\PY@bf=\textbf\def\PY@tc##1{\textcolor[rgb]{0.67,0.36,0.12}{##1}}}
\@namedef{PY@tok@sr}{\def\PY@tc##1{\textcolor[rgb]{0.64,0.35,0.47}{##1}}}
\@namedef{PY@tok@ss}{\def\PY@tc##1{\textcolor[rgb]{0.10,0.09,0.49}{##1}}}
\@namedef{PY@tok@sx}{\def\PY@tc##1{\textcolor[rgb]{0.00,0.50,0.00}{##1}}}
\@namedef{PY@tok@m}{\def\PY@tc##1{\textcolor[rgb]{0.40,0.40,0.40}{##1}}}
\@namedef{PY@tok@gh}{\let\PY@bf=\textbf\def\PY@tc##1{\textcolor[rgb]{0.00,0.00,0.50}{##1}}}
\@namedef{PY@tok@gu}{\let\PY@bf=\textbf\def\PY@tc##1{\textcolor[rgb]{0.50,0.00,0.50}{##1}}}
\@namedef{PY@tok@gd}{\def\PY@tc##1{\textcolor[rgb]{0.63,0.00,0.00}{##1}}}
\@namedef{PY@tok@gi}{\def\PY@tc##1{\textcolor[rgb]{0.00,0.52,0.00}{##1}}}
\@namedef{PY@tok@gr}{\def\PY@tc##1{\textcolor[rgb]{0.89,0.00,0.00}{##1}}}
\@namedef{PY@tok@ge}{\let\PY@it=\textit}
\@namedef{PY@tok@gs}{\let\PY@bf=\textbf}
\@namedef{PY@tok@gp}{\let\PY@bf=\textbf\def\PY@tc##1{\textcolor[rgb]{0.00,0.00,0.50}{##1}}}
\@namedef{PY@tok@go}{\def\PY@tc##1{\textcolor[rgb]{0.44,0.44,0.44}{##1}}}
\@namedef{PY@tok@gt}{\def\PY@tc##1{\textcolor[rgb]{0.00,0.27,0.87}{##1}}}
\@namedef{PY@tok@err}{\def\PY@bc##1{{\setlength{\fboxsep}{\string -\fboxrule}\fcolorbox[rgb]{1.00,0.00,0.00}{1,1,1}{\strut ##1}}}}
\@namedef{PY@tok@kc}{\let\PY@bf=\textbf\def\PY@tc##1{\textcolor[rgb]{0.00,0.50,0.00}{##1}}}
\@namedef{PY@tok@kd}{\let\PY@bf=\textbf\def\PY@tc##1{\textcolor[rgb]{0.00,0.50,0.00}{##1}}}
\@namedef{PY@tok@kn}{\let\PY@bf=\textbf\def\PY@tc##1{\textcolor[rgb]{0.00,0.50,0.00}{##1}}}
\@namedef{PY@tok@kr}{\let\PY@bf=\textbf\def\PY@tc##1{\textcolor[rgb]{0.00,0.50,0.00}{##1}}}
\@namedef{PY@tok@bp}{\def\PY@tc##1{\textcolor[rgb]{0.00,0.50,0.00}{##1}}}
\@namedef{PY@tok@fm}{\def\PY@tc##1{\textcolor[rgb]{0.00,0.00,1.00}{##1}}}
\@namedef{PY@tok@vc}{\def\PY@tc##1{\textcolor[rgb]{0.10,0.09,0.49}{##1}}}
\@namedef{PY@tok@vg}{\def\PY@tc##1{\textcolor[rgb]{0.10,0.09,0.49}{##1}}}
\@namedef{PY@tok@vi}{\def\PY@tc##1{\textcolor[rgb]{0.10,0.09,0.49}{##1}}}
\@namedef{PY@tok@vm}{\def\PY@tc##1{\textcolor[rgb]{0.10,0.09,0.49}{##1}}}
\@namedef{PY@tok@sa}{\def\PY@tc##1{\textcolor[rgb]{0.73,0.13,0.13}{##1}}}
\@namedef{PY@tok@sb}{\def\PY@tc##1{\textcolor[rgb]{0.73,0.13,0.13}{##1}}}
\@namedef{PY@tok@sc}{\def\PY@tc##1{\textcolor[rgb]{0.73,0.13,0.13}{##1}}}
\@namedef{PY@tok@dl}{\def\PY@tc##1{\textcolor[rgb]{0.73,0.13,0.13}{##1}}}
\@namedef{PY@tok@s2}{\def\PY@tc##1{\textcolor[rgb]{0.73,0.13,0.13}{##1}}}
\@namedef{PY@tok@sh}{\def\PY@tc##1{\textcolor[rgb]{0.73,0.13,0.13}{##1}}}
\@namedef{PY@tok@s1}{\def\PY@tc##1{\textcolor[rgb]{0.73,0.13,0.13}{##1}}}
\@namedef{PY@tok@mb}{\def\PY@tc##1{\textcolor[rgb]{0.40,0.40,0.40}{##1}}}
\@namedef{PY@tok@mf}{\def\PY@tc##1{\textcolor[rgb]{0.40,0.40,0.40}{##1}}}
\@namedef{PY@tok@mh}{\def\PY@tc##1{\textcolor[rgb]{0.40,0.40,0.40}{##1}}}
\@namedef{PY@tok@mi}{\def\PY@tc##1{\textcolor[rgb]{0.40,0.40,0.40}{##1}}}
\@namedef{PY@tok@il}{\def\PY@tc##1{\textcolor[rgb]{0.40,0.40,0.40}{##1}}}
\@namedef{PY@tok@mo}{\def\PY@tc##1{\textcolor[rgb]{0.40,0.40,0.40}{##1}}}
\@namedef{PY@tok@ch}{\let\PY@it=\textit\def\PY@tc##1{\textcolor[rgb]{0.24,0.48,0.48}{##1}}}
\@namedef{PY@tok@cm}{\let\PY@it=\textit\def\PY@tc##1{\textcolor[rgb]{0.24,0.48,0.48}{##1}}}
\@namedef{PY@tok@cpf}{\let\PY@it=\textit\def\PY@tc##1{\textcolor[rgb]{0.24,0.48,0.48}{##1}}}
\@namedef{PY@tok@c1}{\let\PY@it=\textit\def\PY@tc##1{\textcolor[rgb]{0.24,0.48,0.48}{##1}}}
\@namedef{PY@tok@cs}{\let\PY@it=\textit\def\PY@tc##1{\textcolor[rgb]{0.24,0.48,0.48}{##1}}}

\def\PYZbs{\char`\\}
\def\PYZus{\char`\_}
\def\PYZob{\char`\{}
\def\PYZcb{\char`\}}
\def\PYZca{\char`\^}
\def\PYZam{\char`\&}
\def\PYZlt{\char`\<}
\def\PYZgt{\char`\>}
\def\PYZsh{\char`\#}
\def\PYZpc{\char`\%}
\def\PYZdl{\char`\$}
\def\PYZhy{\char`\-}
\def\PYZsq{\char`\'}
\def\PYZdq{\char`\"}
\def\PYZti{\char`\~}
% for compatibility with earlier versions
\def\PYZat{@}
\def\PYZlb{[}
\def\PYZrb{]}
\makeatother


    % For linebreaks inside Verbatim environment from package fancyvrb.
    \makeatletter
        \newbox\Wrappedcontinuationbox
        \newbox\Wrappedvisiblespacebox
        \newcommand*\Wrappedvisiblespace {\textcolor{red}{\textvisiblespace}}
        \newcommand*\Wrappedcontinuationsymbol {\textcolor{red}{\llap{\tiny$\m@th\hookrightarrow$}}}
        \newcommand*\Wrappedcontinuationindent {3ex }
        \newcommand*\Wrappedafterbreak {\kern\Wrappedcontinuationindent\copy\Wrappedcontinuationbox}
        % Take advantage of the already applied Pygments mark-up to insert
        % potential linebreaks for TeX processing.
        %        {, <, #, %, $, ' and ": go to next line.
        %        _, }, ^, &, >, - and ~: stay at end of broken line.
        % Use of \textquotesingle for straight quote.
        \newcommand*\Wrappedbreaksatspecials {%
            \def\PYGZus{\discretionary{\char`\_}{\Wrappedafterbreak}{\char`\_}}%
            \def\PYGZob{\discretionary{}{\Wrappedafterbreak\char`\{}{\char`\{}}%
            \def\PYGZcb{\discretionary{\char`\}}{\Wrappedafterbreak}{\char`\}}}%
            \def\PYGZca{\discretionary{\char`\^}{\Wrappedafterbreak}{\char`\^}}%
            \def\PYGZam{\discretionary{\char`\&}{\Wrappedafterbreak}{\char`\&}}%
            \def\PYGZlt{\discretionary{}{\Wrappedafterbreak\char`\<}{\char`\<}}%
            \def\PYGZgt{\discretionary{\char`\>}{\Wrappedafterbreak}{\char`\>}}%
            \def\PYGZsh{\discretionary{}{\Wrappedafterbreak\char`\#}{\char`\#}}%
            \def\PYGZpc{\discretionary{}{\Wrappedafterbreak\char`\%}{\char`\%}}%
            \def\PYGZdl{\discretionary{}{\Wrappedafterbreak\char`\$}{\char`\$}}%
            \def\PYGZhy{\discretionary{\char`\-}{\Wrappedafterbreak}{\char`\-}}%
            \def\PYGZsq{\discretionary{}{\Wrappedafterbreak\textquotesingle}{\textquotesingle}}%
            \def\PYGZdq{\discretionary{}{\Wrappedafterbreak\char`\"}{\char`\"}}%
            \def\PYGZti{\discretionary{\char`\~}{\Wrappedafterbreak}{\char`\~}}%
        }
        % Some characters . , ; ? ! / are not pygmentized.
        % This macro makes them "active" and they will insert potential linebreaks
        \newcommand*\Wrappedbreaksatpunct {%
            \lccode`\~`\.\lowercase{\def~}{\discretionary{\hbox{\char`\.}}{\Wrappedafterbreak}{\hbox{\char`\.}}}%
            \lccode`\~`\,\lowercase{\def~}{\discretionary{\hbox{\char`\,}}{\Wrappedafterbreak}{\hbox{\char`\,}}}%
            \lccode`\~`\;\lowercase{\def~}{\discretionary{\hbox{\char`\;}}{\Wrappedafterbreak}{\hbox{\char`\;}}}%
            \lccode`\~`\:\lowercase{\def~}{\discretionary{\hbox{\char`\:}}{\Wrappedafterbreak}{\hbox{\char`\:}}}%
            \lccode`\~`\?\lowercase{\def~}{\discretionary{\hbox{\char`\?}}{\Wrappedafterbreak}{\hbox{\char`\?}}}%
            \lccode`\~`\!\lowercase{\def~}{\discretionary{\hbox{\char`\!}}{\Wrappedafterbreak}{\hbox{\char`\!}}}%
            \lccode`\~`\/\lowercase{\def~}{\discretionary{\hbox{\char`\/}}{\Wrappedafterbreak}{\hbox{\char`\/}}}%
            \catcode`\.\active
            \catcode`\,\active
            \catcode`\;\active
            \catcode`\:\active
            \catcode`\?\active
            \catcode`\!\active
            \catcode`\/\active
            \lccode`\~`\~
        }
    \makeatother

    \let\OriginalVerbatim=\Verbatim
    \makeatletter
    \renewcommand{\Verbatim}[1][1]{%
        %\parskip\z@skip
        \sbox\Wrappedcontinuationbox {\Wrappedcontinuationsymbol}%
        \sbox\Wrappedvisiblespacebox {\FV@SetupFont\Wrappedvisiblespace}%
        \def\FancyVerbFormatLine ##1{\hsize\linewidth
            \vtop{\raggedright\hyphenpenalty\z@\exhyphenpenalty\z@
                \doublehyphendemerits\z@\finalhyphendemerits\z@
                \strut ##1\strut}%
        }%
        % If the linebreak is at a space, the latter will be displayed as visible
        % space at end of first line, and a continuation symbol starts next line.
        % Stretch/shrink are however usually zero for typewriter font.
        \def\FV@Space {%
            \nobreak\hskip\z@ plus\fontdimen3\font minus\fontdimen4\font
            \discretionary{\copy\Wrappedvisiblespacebox}{\Wrappedafterbreak}
            {\kern\fontdimen2\font}%
        }%

        % Allow breaks at special characters using \PYG... macros.
        \Wrappedbreaksatspecials
        % Breaks at punctuation characters . , ; ? ! and / need catcode=\active
        \OriginalVerbatim[#1,codes*=\Wrappedbreaksatpunct]%
    }
    \makeatother

    % Exact colors from NB
    \definecolor{incolor}{HTML}{303F9F}
    \definecolor{outcolor}{HTML}{D84315}
    \definecolor{cellborder}{HTML}{CFCFCF}
    \definecolor{cellbackground}{HTML}{F7F7F7}

    % prompt
    \makeatletter
    \newcommand{\boxspacing}{\kern\kvtcb@left@rule\kern\kvtcb@boxsep}
    \makeatother
    \newcommand{\prompt}[4]{
        {\ttfamily\llap{{\color{#2}[#3]:\hspace{3pt}#4}}\vspace{-\baselineskip}}
    }
    

    
    % Prevent overflowing lines due to hard-to-break entities
    \sloppy
    % Setup hyperref package
    \hypersetup{
      breaklinks=true,  % so long urls are correctly broken across lines
      colorlinks=true,
      urlcolor=urlcolor,
      linkcolor=linkcolor,
      citecolor=citecolor,
      }
    % Slightly bigger margins than the latex defaults
    
    \geometry{verbose,tmargin=1in,bmargin=1in,lmargin=1in,rmargin=1in}
    
    

\begin{document}
    
    \maketitle
    
    

    
    \section{Quantum Linear Algebra}\label{quantum-linear-algebra}

The Harrow--Hassidim--Lloyd (HHL) algorithm {[}Project 1{]}\\
\emph{By Sonny Lowe, David Lee, Arav Raval}

    This notebook will discuss the Harrow-Hassidim-Lloyd (HHL) quantum
algorithm, meant for solving a linear system
\[A\vec{x}=\vec{b} \text{ ,  where } A \text{ is a hermitian matrix}\]
and where \(\vec{x}\) and \(\vec{b}\) ultimately represent quantum
states \(\ket{x}\) and \(\ket{b}\) respectively. We will provide a
derivation, implementation, generalization to non-hermitian matrices, as
well as the context for HHL as a subroutine.

Note: we will generally be working in the normalized domain.

    \subsubsection{Section 1: Mathematical
Derivation}\label{section-1-mathematical-derivation}

Our problem is represented as \[A\ket{x} = \ket{b}\] where
\(\ket{b} \in \mathbb{C}^N\) is some given quantum state and
\(A \in \mathbb{C}^{N\times N}\). Our goal is to solve for
\(\ket{x} \in \mathbb{C}^N\) under a few conditions: - \(A\) is
hermitian such that \(A = A^\dagger\) - \(A\) is \(s\)-sparse and well
conditioned, meaning it has at most \(s\) nonzero entries per row and
its condition number \(\kappa(A)\) is relatively small such that the
system is stable and less sensitive to perturbations. We will break down
the derivation into several steps. - We have access to an ``oracle'' of
A in that we have its eigenvalues to use in our circuit (potential
weakness). - We assume there exists some efficient algorithm to prepare
\(\ket{b}\)

\paragraph{\texorpdfstring{\textbf{1. Rewriting and
Initialization}}{1. Rewriting and Initialization}}\label{rewriting-and-initialization}

Given that \(A\) is a hermitian matrix, there exists a spectral
decomposition such that the matrix can be diagonalized by unitary
transformations.
\[A = UDU^T,\quad \text{ where } U \text{ is a unitary matrix and } D \text{ is diagonal composed of the real eigenvalues of } A\]
Since the columns of \(U\) form an orthonormal basis and are the
eigenvalues, we can rewrite this decomposition where \(\ket{u_{j}}\) is
the \(j^{th}\) eigenvector of \(A\) with respective eigenvalue
\(\lambda_{j}\) as:
\[A = \sum_{i=0}^{N-1}\lambda_{i}\ket{u_{i}}\bra{u_{i}}, \quad \lambda_{i}\in\mathbb{ R }\]
Likewise, as a linear transformation, we can write our resultant vector
\(\ket{b}\) in the eigenbasis of \(A\).
\[\ket{b} = \sum_{i=0}^{N-1}b_{i}\ket{u_{i}}, \quad b_{i}\in\mathbb{ C }\]
Thus, our problem can now be rewritten as:
\[\ket{x} = A^{-1}\ket{b} = \sum_{i=0}^{N-1}\frac{1}{\lambda_{i}}b_{i}\ket{u_{i}}\bra{u_{i}}u_{i}\rangle = \sum_{i=0}^{N-1}\frac{1}{\lambda_{i}}b_{i}\ket{u_{i}}\]

Moreover, note that to represent a state vector \(\ket{b}\) of dimension
\(N\), to prepare it as a quantum register, we prepare it in binary
basis using \(\log_2{N}\) qubits. Therefore, from here on,
\(N = \log_2{N}\).

\paragraph{\texorpdfstring{\textbf{2. Quantum Phase
Estimation}}{2. Quantum Phase Estimation}}\label{quantum-phase-estimation}

The problem first boils down to being able to decompose a matrix to find
its eigenvalues and eigenvectors in a computationally efficient manner.
At a high level, Quantum Phase Estimation is a procedure that performs a
series of controlled-\(U\) gates given some unitary matrix \(U\) with
eigenvalues of the form \(e^{2\pi i \theta}\), and finds the phase
\(\theta\).

The number of qubits required for QPE is defined by the required
additive error epsilon \(\epsilon\), where we will set
\(n_q = O(\log{\frac{1}{\epsilon}})\) representing the number of qubits
needed.

For HHL, first, we start with two registers \(\ket{0}^{\otimes n_q}\)
and \(\ket{b}\). Our initial state is thus
\(\ket{\Phi_0} = \ket{0}^{\otimes n_q} \otimes \ket{b}.\) Then we can
characterize the behavior of \textbf{QPE} as:
\[\textbf{QPE}(U,\ket{0}^{\otimes n_q},\ket{b}) = \textbf{QPE}(U,\ket{0}^{\otimes n_q},\sum_{j=0}^{N-1}b_{j}\ket{u_{j}}) = \sum_{j=0}^{N-1}b_{j} \ket{\tilde{\lambda_{j}}}_{n_q}  \ket{u_{j}}, \quad\text{ for some clever choice of } U\]

We will choose \(U = e^{iAt}\) for some constant \(t\) such that \(U\)
is governed by the same eigenvalues of \(A\) and we have that, where
\(\ket{u_{j}}\) is the \(j^{th}\) eigenvector of \(A\) with respective
eigenvalue \(\lambda_{j}\):
\[U = e^{iAt} = \sum_{j=0}^{N-1}e^{i\lambda_{j}t}\ket{u_j}\bra{u_j}\]

\begin{quote}
\begin{enumerate}
\def\labelenumi{\arabic{enumi}.}
\tightlist
\item
  First, we hadamard our first register of \(\ket{0}^{\otimes n_q}\).
  \(H^{\otimes n_q}\ket{\Phi_0} = \frac{1}{2^{n/2}}(\ket{0} + \ket{1})^{\otimes n_q} \otimes \ket{b}\)
\item
  Next, we perform a series of \(n_q\) controlled-\(U\) gates such that
  for \(k = {0,...,2^{n_q-1}}\), we create the gate (where \(\ket{k}\)
  is the k-th qubit in the \(n_q\) registrar and t is some evolution
  time):
  \(CU^{2^k}(\ket{k}, \ket{b}) =_{\text{if} \ket{k}=\ket{1}} \ket{k} \otimes U^{2^k}\ket{b}\).
  The composition of all N of these such gates yields the state (with
  normalization):
  \[ \ket{b} \otimes \frac{1}{\sqrt{2^n}} \sum_{k=0}^{2^{n_q-1}} e^{i k \lambda_j t} \ket{k}\]
\item
  Consider just our \(n_q\) (control) register, which is now superposed
  through unitary entanglements to the state:
  \[ \frac{1}{\sqrt{2^n}} \sum_{k=0}^{2^{n_q-1}} e^{i k \lambda_j t} \ket{k} \]
  \[ \text{where the phase factor is }\quad \lambda_i t = 2\pi\theta \]
  It is clear now that the encoded phase factor we desire (which
  represents the eigenvalues of \(A\)), can be extracted through the
  inverse quantum fourier transform operation. Thus, applying
  \textbf{QFT\(^{-1}\)} has the following behavior:
  \[\textbf{QTF}^{-1}(\frac{1}{\sqrt{2^n}} \sum_{k=0}^{2^{n_q-1}} e^{i k \lambda_j t} \ket{k}) \mapsto \ket{\theta 2^{n_q}} \quad\text{ which encodes } \theta_j 2^{n_q} = \lambda_j t / 2\pi \quad\Rightarrow \tilde{\lambda_j} = \frac{\lambda_j t 2^{n_q}}{2\pi}\]
  Thus, we have essentially calculated an approximation of our actual
  eigenvalues. Thus, for our total state, we are left with a register
  that by superposition, now encodes an approximation of our eigenvalues
  into \(n_q\) qubits.:
  \[\sum_{j=0}^{N-1}b_{j}\ket{\tilde{\lambda_j}}_{n_q}\ket{u_j}\]
\end{enumerate}
\end{quote}

\paragraph{\texorpdfstring{\textbf{3. Controlled
Rotation}}{3. Controlled Rotation}}\label{controlled-rotation}

With our new intermediate state
\(\sum_{j=0}^{N-1}b_{j}\ket{\tilde{\lambda_j}}_{n_q}\ket{u_j}\), our
next problem is to produce the ``eigenvalues' reciprocal.'' We will do
this by applying a series of controlled rotations. These rotations will
utilize the control registers from QPE which now encodes the eigenvalues
in binary with \(n_q\) bits. All rotations will be applied onto a new
ancilla register that will aid in extracting
\(\frac{1}{\tilde{\lambda}}\). (Checkpoint: by now, we have created 3
registers with a total state of
\(\ket{\tilde{\lambda}}_{n_q} \otimes \ket{b}_N \otimes \ket{0}_{ancilla}\))

To do this, we will encode the ancilla qubit to the following state:
\[\ket{0}_{ancilla} \mapsto \sum_{j=0}^{N-1} \sqrt{1 - \left(\frac{C}{\tilde{\lambda_j}}\right)^2}\ket{0}+\frac{C}{\tilde{\lambda_j}}\ket{1}\]

\(C\) is some constant that corresponds to the success rate of the
ancilla being in the state \(\ket{1}\). However, it must not exceed the
value of the smallest \(\tilde{\lambda_j}\) as evident. Therefore, we
aim to find the maximal \(C\) such that
\(C < \min_{j}\tilde{\lambda_j}\).

Considering the general state
\(\ket{\psi} = \cos(\theta/2)\ket{0} + e^{i\phi}\sin(\theta/2)\ket{1}\)
allows us to clearly see that
\(\theta_j = 2\arcsin(C/\tilde{\lambda_j}) \quad \forall j\). Now, for
each angle, we apply the rotation using a multi-controlled gate that
triggers if the control QPE register has the encoding of the
\(\tilde{\lambda_j}\) eigenvalue. We know that our reciprocal circuit
has succeeded if the ancilla bit is in the state \(\ket{1}\).

Repeated until success, we are left with
\[\sum_{j=0}^{N-1}\ket{\tilde{\lambda_j}}\otimes b_{j}\ket{u_j} \otimes \sqrt{1 - \left(\frac{C}{\tilde{\lambda_j}}\right)^2}\ket{0}+\frac{C}{\tilde{\lambda_j}}\ket{1}\]

\paragraph{\texorpdfstring{\textbf{4.
Uncompute}}{4. Uncompute}}\label{uncompute}

Next, we want to uncompute our eigenvalue register by applying the
dagger of QPE. This consists of first applying QFT, then simplying
applying the conditioned inverse of the unitary matrix \(U\) followed by
hadamards. We are left with a scaled approximation:
\[\sum_{j=0}^{N-1}\ket{0}\otimes b_{j}\ket{u_j} \otimes \sqrt{1 - \left(\frac{C}{\tilde{\lambda_j}}\right)^2}\ket{0}+\frac{C}{\lambda_j}\ket{1}\]

\paragraph{\texorpdfstring{\textbf{5. Measure
Ancillas}}{5. Measure Ancillas}}\label{measure-ancillas}

Finally, we will measure the ancilla, which after normalization, yields
the post measurement state if our ancilla has the desired outcome of
\(\ket{1}\) in the form:
\[\sum_{j=0}^{N-1}\frac{1}{\lambda_{j}}b_{j}\ket{u_{j}}\ket{1}_{ancilla}\]
It is here that we see that this process is not deterministic and can
fail up to a probability if our ancilla is in \(\ket{0}\). Thus, we will
repeat the process of steps 2-4 until success (ancilla is \(\ket{1}\)).

\paragraph{\texorpdfstring{\textbf{6. Extracting
Answer}}{6. Extracting Answer}}\label{extracting-answer}

We can see now that our registers hold our solution in the form of
\(A^{-1}\ket{b}\). However, since the state \(\ket{x}\) is determined in
our registers as a superposition, we cannot measure the full state in
one shot without collapsing it to a single basis state. However, by
measuring certain observables, you can gain useful information about the
state without fully collapsing it. The quantum expectation value
\(\bra{x}M\ket{x}\) for a given observable \(M\) is useful in the
context of a subroutine because it can allow us to: - obtain information
about the solution \(\ket{x}\) for instance, the probability
distribution of components of \(\ket{x}\) in a particular basis. -
Measure a function of the solution, such as a dot product or a norm,
which might be useful in applications without needing the full solution.

\paragraph{\texorpdfstring{\textbf{Our overall circuit
diagram:}}{Our overall circuit diagram:}}\label{our-overall-circuit-diagram}

    \subsubsection{Section 2: Generalization to
Non-Hermitian}\label{section-2-generalization-to-non-hermitian}

The case where \(A \in \mathbb{C}^{N\times N}\) is not hermitian is
actually quite easy to resolve. Simply construct a new matrix
\(C \in \mathbb{C}^{2N \times 2N}\):
\[ C = \begin{bmatrix}0&A\\A^{\dagger}&0\end{bmatrix} \] Notice that
\(C\) is immediately hermitian:
\[ C^{\dagger} = \begin{bmatrix}0&A^{\dagger}\\A^{\dagger^{\dagger}}&0\end{bmatrix}^T = \begin{bmatrix}0&A\\A^{\dagger}&0\end{bmatrix} \]
and the properties of sparesness and well conditioning remain relative
to this new matrix.

Notice now that our problem changes as such:
\[A\vec{x}=\vec{b} \text{ ,  where } A \text{ is non-hermitian} \quad\Rightarrow\quad C\begin{bmatrix}0\\x\end{bmatrix} = \begin{bmatrix}b\\0\end{bmatrix} \text{ ,  where } C \text{ is a hermitian matrix} \]

    \subsubsection{Section 3: Sample
Implementation}\label{section-3-sample-implementation}

This next section contains our implementation of HHL tested on a 3x3
invertible hermitian matrix selected to be sparse and relatively stable
(low condition number). This implementation takes a step further than
many tutorials online by removing the need for manual tinkering during
the creation of the circuit. It is entirely self-contained as a hybrid
algorithm.

This project utilizes IBM's Qiskit - qiskit == 0.44.2 - qiskit-aer ==
0.12.2 - qiskit-terra == 0.25.2.1

    \begin{tcolorbox}[breakable, size=fbox, boxrule=1pt, pad at break*=1mm,colback=cellbackground, colframe=cellborder]
\prompt{In}{incolor}{1}{\boxspacing}
\begin{Verbatim}[commandchars=\\\{\}]
\PY{c+c1}{\PYZsh{} Numerical Analysis Imports and Qiskit Imports}
\PY{k+kn}{import} \PY{n+nn}{numpy} \PY{k}{as} \PY{n+nn}{np}
\PY{k+kn}{import} \PY{n+nn}{scipy}
\PY{k+kn}{from} \PY{n+nn}{numpy} \PY{k+kn}{import} \PY{n}{pi}\PY{p}{,}\PY{n}{sqrt}

\PY{k+kn}{from} \PY{n+nn}{qiskit} \PY{k+kn}{import}\PY{p}{(}\PY{n}{QuantumCircuit}\PY{p}{,} \PY{n}{execute}\PY{p}{,} \PY{n}{Aer}\PY{p}{,} \PY{n}{ClassicalRegister}\PY{p}{,} \PY{n}{QuantumRegister}\PY{p}{)}
\PY{n}{backend\PYZus{}svec} \PY{o}{=} \PY{n}{Aer}\PY{o}{.}\PY{n}{get\PYZus{}backend}\PY{p}{(}\PY{l+s+s1}{\PYZsq{}}\PY{l+s+s1}{statevector\PYZus{}simulator}\PY{l+s+s1}{\PYZsq{}}\PY{p}{)}
\PY{n}{backend\PYZus{}qasm} \PY{o}{=} \PY{n}{Aer}\PY{o}{.}\PY{n}{get\PYZus{}backend}\PY{p}{(}\PY{l+s+s1}{\PYZsq{}}\PY{l+s+s1}{qasm\PYZus{}simulator}\PY{l+s+s1}{\PYZsq{}}\PY{p}{)}

\PY{k+kn}{from} \PY{n+nn}{qiskit}\PY{n+nn}{.}\PY{n+nn}{extensions} \PY{k+kn}{import} \PY{n}{UnitaryGate}
\PY{k+kn}{from} \PY{n+nn}{qiskit}\PY{n+nn}{.}\PY{n+nn}{circuit}\PY{n+nn}{.}\PY{n+nn}{library}\PY{n+nn}{.}\PY{n+nn}{standard\PYZus{}gates} \PY{k+kn}{import} \PY{n}{UGate}

\PY{c+c1}{\PYZsh{} Visualizations}
\PY{k+kn}{from} \PY{n+nn}{qiskit}\PY{n+nn}{.}\PY{n+nn}{visualization} \PY{k+kn}{import} \PY{n}{plot\PYZus{}histogram}
\PY{k+kn}{import} \PY{n+nn}{matplotlib}\PY{n+nn}{.}\PY{n+nn}{pyplot} \PY{k}{as} \PY{n+nn}{plt}
\end{Verbatim}
\end{tcolorbox}

    Given a Matrix, our code will check certain constraints. The most
important of the modifications it will make regards whether or not the
given matrix is hermitian. If not, it will create the block matrix
outlined in Section 2. In addition, it will pad the vector and matrix as
necessary such that it can be represented in qubits (its dimension is a
power of 2).

Some additional functions: eigenvalue scaling \((-1,1)\), unitary matrix
\(U = e^{iAt}\) creation.

    \begin{tcolorbox}[breakable, size=fbox, boxrule=1pt, pad at break*=1mm,colback=cellbackground, colframe=cellborder]
\prompt{In}{incolor}{2}{\boxspacing}
\begin{Verbatim}[commandchars=\\\{\}]
\PY{c+c1}{\PYZsh{}\PYZsh{}\PYZsh{}\PYZsh{}\PYZsh{}\PYZsh{}\PYZsh{}\PYZsh{}\PYZsh{}\PYZsh{}\PYZsh{}\PYZsh{}\PYZsh{}\PYZsh{}\PYZsh{}\PYZsh{}\PYZsh{}\PYZsh{}\PYZsh{}\PYZsh{}\PYZsh{}\PYZsh{}\PYZsh{}\PYZsh{}\PYZsh{}\PYZsh{}\PYZsh{}\PYZsh{}\PYZsh{}\PYZsh{}\PYZsh{}\PYZsh{}\PYZsh{}\PYZsh{}\PYZsh{}\PYZsh{}\PYZsh{}\PYZsh{}\PYZsh{}\PYZsh{}\PYZsh{}\PYZsh{}\PYZsh{}\PYZsh{}\PYZsh{}\PYZsh{}\PYZsh{}\PYZsh{}\PYZsh{}\PYZsh{}\PYZsh{}\PYZsh{}\PYZsh{}\PYZsh{}}
\PY{c+c1}{\PYZsh{}\PYZsh{}\PYZsh{}\PYZsh{}\PYZsh{}\PYZsh{}\PYZsh{}\PYZsh{}\PYZsh{}\PYZsh{}\PYZsh{}\PYZsh{} Matrix and Vector Handling \PYZsh{}\PYZsh{}\PYZsh{}\PYZsh{}\PYZsh{}\PYZsh{}\PYZsh{}\PYZsh{}\PYZsh{}\PYZsh{}\PYZsh{}\PYZsh{}\PYZsh{}\PYZsh{}}
\PY{c+c1}{\PYZsh{}\PYZsh{}\PYZsh{}\PYZsh{}\PYZsh{}\PYZsh{}\PYZsh{}\PYZsh{}\PYZsh{}\PYZsh{}\PYZsh{}\PYZsh{}\PYZsh{}\PYZsh{}\PYZsh{}\PYZsh{}\PYZsh{}\PYZsh{}\PYZsh{}\PYZsh{}\PYZsh{}\PYZsh{}\PYZsh{}\PYZsh{}\PYZsh{}\PYZsh{}\PYZsh{}\PYZsh{}\PYZsh{}\PYZsh{}\PYZsh{}\PYZsh{}\PYZsh{}\PYZsh{}\PYZsh{}\PYZsh{}\PYZsh{}\PYZsh{}\PYZsh{}\PYZsh{}\PYZsh{}\PYZsh{}\PYZsh{}\PYZsh{}\PYZsh{}\PYZsh{}\PYZsh{}\PYZsh{}\PYZsh{}\PYZsh{}\PYZsh{}\PYZsh{}\PYZsh{}\PYZsh{}}
\PY{k}{def} \PY{n+nf}{prepare\PYZus{}hhlparams}\PY{p}{(}
        \PY{n}{A}\PY{p}{:} \PY{n}{np}\PY{o}{.}\PY{n}{ndarray}\PY{p}{,}
        \PY{n}{b}\PY{p}{:} \PY{n}{np}\PY{o}{.}\PY{n}{ndarray}\PY{p}{,}
        \PY{n}{tol}\PY{p}{:} \PY{n+nb}{float}\PY{p}{,}
        \PY{n}{ev\PYZus{}time}\PY{p}{:} \PY{n+nb}{float}\PY{p}{,}
\PY{p}{)}\PY{o}{\PYZhy{}}\PY{o}{\PYZgt{}} \PY{p}{(}\PY{n}{np}\PY{o}{.}\PY{n}{ndarray}\PY{p}{,} \PY{n}{np}\PY{o}{.}\PY{n}{ndarray}\PY{p}{,} \PY{n}{np}\PY{o}{.}\PY{n}{ndarray}\PY{p}{,} \PY{n+nb}{int}\PY{p}{,} \PY{n+nb}{int}\PY{p}{)}\PY{p}{:}
    \PY{k}{if} \PY{o+ow}{not} \PY{n+nb}{isinstance}\PY{p}{(}\PY{n}{A}\PY{p}{,} \PY{n}{np}\PY{o}{.}\PY{n}{ndarray}\PY{p}{)} \PY{o+ow}{or} \PY{n}{A}\PY{o}{.}\PY{n}{ndim} \PY{o}{!=} \PY{l+m+mi}{2}\PY{p}{:}
        \PY{k}{raise} \PY{n+ne}{TypeError}\PY{p}{(}\PY{l+s+s1}{\PYZsq{}}\PY{l+s+s1}{A must be a 2D numpy matrix}\PY{l+s+s1}{\PYZsq{}}\PY{p}{)}
    \PY{k}{elif} \PY{o+ow}{not} \PY{n+nb}{isinstance}\PY{p}{(}\PY{n}{b}\PY{p}{,} \PY{n}{np}\PY{o}{.}\PY{n}{ndarray}\PY{p}{)} \PY{o+ow}{or} \PY{n}{b}\PY{o}{.}\PY{n}{ndim} \PY{o}{!=} \PY{l+m+mi}{1}\PY{p}{:}
        \PY{k}{raise} \PY{n+ne}{TypeError}\PY{p}{(}\PY{l+s+s1}{\PYZsq{}}\PY{l+s+s1}{b must be a 1D numpy matrix}\PY{l+s+s1}{\PYZsq{}}\PY{p}{)}
    \PY{k}{elif} \PY{n}{A}\PY{o}{.}\PY{n}{shape}\PY{p}{[}\PY{l+m+mi}{0}\PY{p}{]} \PY{o}{!=} \PY{n}{A}\PY{o}{.}\PY{n}{shape}\PY{p}{[}\PY{l+m+mi}{1}\PY{p}{]}\PY{p}{:}
        \PY{k}{raise} \PY{n+ne}{ValueError}\PY{p}{(}\PY{l+s+s2}{\PYZdq{}}\PY{l+s+s2}{A must be a square matrix.}\PY{l+s+s2}{\PYZdq{}}\PY{p}{)}
    \PY{k}{elif} \PY{n}{A}\PY{o}{.}\PY{n}{shape}\PY{p}{[}\PY{l+m+mi}{0}\PY{p}{]} \PY{o}{!=} \PY{n}{b}\PY{o}{.}\PY{n}{size}\PY{p}{:} 
        \PY{k}{raise} \PY{n+ne}{ValueError}\PY{p}{(}\PY{l+s+s2}{\PYZdq{}}\PY{l+s+s2}{Invalid dimensions for linear system.}\PY{l+s+s2}{\PYZdq{}}\PY{p}{)}
    \PY{k}{elif} \PY{n}{np}\PY{o}{.}\PY{n}{linalg}\PY{o}{.}\PY{n}{det}\PY{p}{(}\PY{n}{A}\PY{p}{)} \PY{o}{==} \PY{l+m+mi}{0}\PY{p}{:}
        \PY{k}{raise} \PY{n+ne}{ValueError}\PY{p}{(}\PY{l+s+s2}{\PYZdq{}}\PY{l+s+s2}{A must an invertible matrix.}\PY{l+s+s2}{\PYZdq{}}\PY{p}{)}
    \PY{c+c1}{\PYZsh{} assuming matrix sparsity and well conditioned}
    
    \PY{c+c1}{\PYZsh{}\PYZsh{}\PYZsh{}\PYZsh{} CONSTANTS \PYZsh{}\PYZsh{}\PYZsh{}\PYZsh{}}
    \PY{n}{n} \PY{o}{=} \PY{n+nb}{len}\PY{p}{(}\PY{n}{b}\PY{p}{)}
    \PY{n}{nq} \PY{o}{=} \PY{n+nb}{int}\PY{p}{(}\PY{n}{np}\PY{o}{.}\PY{n}{log}\PY{p}{(}\PY{l+m+mi}{1}\PY{o}{/}\PY{n}{tol}\PY{p}{)}\PY{p}{)} \PY{c+c1}{\PYZsh{} bits needed for QPE register}
    \PY{n}{t} \PY{o}{=} \PY{n}{ev\PYZus{}time} \PY{c+c1}{\PYZsh{} evolution time, multiple of pi}
    
    \PY{c+c1}{\PYZsh{}\PYZsh{}\PYZsh{}\PYZsh{} PREPARE MATRIX \PYZsh{}\PYZsh{}\PYZsh{}\PYZsh{}}
       \PY{c+c1}{\PYZsh{} HERMITIAN CHECKER \PYZsh{}}
    \PY{n}{C} \PY{o}{=} \PY{n}{A}
    \PY{k}{if} \PY{o+ow}{not} \PY{n}{np}\PY{o}{.}\PY{n}{allclose}\PY{p}{(}\PY{n}{A}\PY{p}{,} \PY{n}{np}\PY{o}{.}\PY{n}{conjugate}\PY{p}{(}\PY{n}{A}\PY{o}{.}\PY{n}{T}\PY{p}{)}\PY{p}{)}\PY{p}{:}
        \PY{n}{C} \PY{o}{=} \PY{n}{np}\PY{o}{.}\PY{n}{block}\PY{p}{(}\PY{p}{[}\PY{p}{[}\PY{n}{np}\PY{o}{.}\PY{n}{zeros\PYZus{}like}\PY{p}{(}\PY{n}{A}\PY{p}{)}\PY{p}{,} \PY{n}{A}\PY{p}{]}\PY{p}{,} \PY{p}{[}\PY{n}{np}\PY{o}{.}\PY{n}{conj}\PY{p}{(}\PY{n}{A}\PY{o}{.}\PY{n}{T}\PY{p}{)}\PY{p}{,} \PY{n}{np}\PY{o}{.}\PY{n}{zeros\PYZus{}like}\PY{p}{(}\PY{n}{A}\PY{p}{)}\PY{p}{]}\PY{p}{]}\PY{p}{)}
        \PY{n}{n} \PY{o}{*}\PY{o}{=} \PY{l+m+mi}{2}
       \PY{c+c1}{\PYZsh{} PADDING MATRIX \PYZsh{}}
    \PY{n}{nb} \PY{o}{=} \PY{n+nb}{int}\PY{p}{(}\PY{n}{np}\PY{o}{.}\PY{n}{ceil}\PY{p}{(}\PY{n}{np}\PY{o}{.}\PY{n}{log2}\PY{p}{(}\PY{n}{C}\PY{o}{.}\PY{n}{shape}\PY{p}{[}\PY{l+m+mi}{0}\PY{p}{]}\PY{p}{)}\PY{p}{)}\PY{p}{)}
    \PY{n}{N} \PY{o}{=} \PY{l+m+mi}{2}\PY{o}{*}\PY{o}{*}\PY{n}{nb}
    \PY{k}{if} \PY{n}{C}\PY{o}{.}\PY{n}{shape}\PY{p}{[}\PY{l+m+mi}{0}\PY{p}{]} \PY{o}{!=} \PY{n}{N}\PY{p}{:}
        \PY{n}{C\PYZus{}pad} \PY{o}{=} \PY{n}{np}\PY{o}{.}\PY{n}{eye}\PY{p}{(}\PY{n}{N}\PY{p}{,} \PY{n}{dtype}\PY{o}{=}\PY{n+nb}{complex}\PY{p}{)}
        \PY{n}{C\PYZus{}pad}\PY{p}{[}\PY{p}{:}\PY{n}{n}\PY{p}{,} \PY{p}{:}\PY{n}{n}\PY{p}{]} \PY{o}{=} \PY{n}{C}
        \PY{n}{C} \PY{o}{=} \PY{n}{C\PYZus{}pad}
        
    \PY{n}{eigenvalues}\PY{p}{,} \PY{n}{eigenvectors} \PY{o}{=} \PY{n}{np}\PY{o}{.}\PY{n}{linalg}\PY{o}{.}\PY{n}{eigh}\PY{p}{(}\PY{n}{C}\PY{p}{)}
    \PY{n}{C} \PY{o}{/}\PY{o}{=} \PY{n+nb}{max}\PY{p}{(}\PY{n}{np}\PY{o}{.}\PY{n}{real}\PY{p}{(}\PY{n}{eigenvalues}\PY{p}{)}\PY{p}{)}
    
    \PY{c+c1}{\PYZsh{}\PYZsh{}\PYZsh{}\PYZsh{} PREPARE VECTOR (padding) \PYZsh{}\PYZsh{}\PYZsh{}\PYZsh{}}
    \PY{n}{b\PYZus{}pad} \PY{o}{=} \PY{n}{b}
    \PY{k}{if} \PY{n}{b}\PY{o}{.}\PY{n}{shape}\PY{p}{[}\PY{l+m+mi}{0}\PY{p}{]} \PY{o}{!=} \PY{n}{N}\PY{p}{:}
        \PY{n}{b\PYZus{}pad} \PY{o}{=} \PY{n}{np}\PY{o}{.}\PY{n}{pad}\PY{p}{(}\PY{n}{b}\PY{p}{,} \PY{p}{(}\PY{l+m+mi}{0}\PY{p}{,} \PY{n}{N} \PY{o}{\PYZhy{}} \PY{n+nb}{len}\PY{p}{(}\PY{n}{b}\PY{p}{)}\PY{p}{)}\PY{p}{)}
        
    \PY{c+c1}{\PYZsh{}\PYZsh{}\PYZsh{}\PYZsh{} PREPARE UNITARY MATRIX U \PYZsh{}\PYZsh{}\PYZsh{}\PYZsh{}}
    \PY{n}{eigenvalues}\PY{p}{,} \PY{n}{eigenvectors} \PY{o}{=} \PY{n}{np}\PY{o}{.}\PY{n}{linalg}\PY{o}{.}\PY{n}{eigh}\PY{p}{(}\PY{n}{C}\PY{p}{)}
    \PY{n}{U} \PY{o}{=} \PY{n}{scipy}\PY{o}{.}\PY{n}{linalg}\PY{o}{.}\PY{n}{expm}\PY{p}{(}\PY{l+m+mi}{1}\PY{n}{j}\PY{o}{*}\PY{n}{C}\PY{o}{*}\PY{n}{t}\PY{p}{)}
    
    \PY{n+nb}{print}\PY{p}{(}\PY{l+s+s1}{\PYZsq{}}\PY{l+s+s1}{Eigenvalues of A}\PY{l+s+se}{\PYZbs{}\PYZsq{}}\PY{l+s+s1}{:}\PY{l+s+s1}{\PYZsq{}}\PY{p}{,} \PY{n}{np}\PY{o}{.}\PY{n}{array2string}\PY{p}{(}\PY{n}{eigenvalues}\PY{p}{,} \PY{n}{formatter}\PY{o}{=}\PY{p}{\PYZob{}}\PY{l+s+s1}{\PYZsq{}}\PY{l+s+s1}{float\PYZus{}kind}\PY{l+s+s1}{\PYZsq{}}\PY{p}{:} \PY{k}{lambda} \PY{n}{x}\PY{p}{:} \PY{l+s+sa}{f}\PY{l+s+s2}{\PYZdq{}}\PY{l+s+si}{\PYZob{}}\PY{n}{x}\PY{l+s+si}{:}\PY{l+s+s2}{.3f}\PY{l+s+si}{\PYZcb{}}\PY{l+s+s2}{\PYZdq{}}\PY{p}{\PYZcb{}}\PY{p}{)}\PY{p}{)}
    \PY{n+nb}{print}\PY{p}{(}\PY{l+s+s1}{\PYZsq{}}\PY{l+s+s1}{Condition number of A}\PY{l+s+se}{\PYZbs{}\PYZsq{}}\PY{l+s+s1}{:}\PY{l+s+s1}{\PYZsq{}}\PY{p}{,} \PY{n}{np}\PY{o}{.}\PY{n}{linalg}\PY{o}{.}\PY{n}{cond}\PY{p}{(}\PY{n}{A}\PY{p}{)}\PY{p}{)}
    \PY{k}{return} \PY{n}{C}\PY{p}{,} \PY{n}{b\PYZus{}pad}\PY{p}{,} \PY{n}{U}\PY{p}{,} \PY{n}{nb}\PY{p}{,} \PY{n}{nq}
    
\end{Verbatim}
\end{tcolorbox}

    The helper functions below will help us analyze our results at the very
end by comparing the amplitudes/probability distribution of our HHL
derived answer and a classically computed answer.

    \begin{tcolorbox}[breakable, size=fbox, boxrule=1pt, pad at break*=1mm,colback=cellbackground, colframe=cellborder]
\prompt{In}{incolor}{3}{\boxspacing}
\begin{Verbatim}[commandchars=\\\{\}]
\PY{c+c1}{\PYZsh{}\PYZsh{}\PYZsh{}\PYZsh{}\PYZsh{}\PYZsh{}\PYZsh{}\PYZsh{}\PYZsh{}\PYZsh{}\PYZsh{}\PYZsh{}\PYZsh{}\PYZsh{}\PYZsh{}\PYZsh{}\PYZsh{}\PYZsh{}\PYZsh{}\PYZsh{}\PYZsh{}\PYZsh{}\PYZsh{}\PYZsh{}\PYZsh{}\PYZsh{}\PYZsh{}\PYZsh{}\PYZsh{}\PYZsh{}\PYZsh{}\PYZsh{}\PYZsh{}\PYZsh{}\PYZsh{}\PYZsh{}\PYZsh{}\PYZsh{}\PYZsh{}\PYZsh{}\PYZsh{}\PYZsh{}\PYZsh{}\PYZsh{}\PYZsh{}\PYZsh{}\PYZsh{}\PYZsh{}\PYZsh{}\PYZsh{}\PYZsh{}\PYZsh{}\PYZsh{}\PYZsh{}}
\PY{c+c1}{\PYZsh{}\PYZsh{}\PYZsh{}\PYZsh{}\PYZsh{}\PYZsh{}\PYZsh{}\PYZsh{}\PYZsh{}\PYZsh{}\PYZsh{}\PYZsh{}\PYZsh{}\PYZsh{}\PYZsh{}\PYZsh{} Results Analyzers \PYZsh{}\PYZsh{}\PYZsh{}\PYZsh{}\PYZsh{}\PYZsh{}\PYZsh{}\PYZsh{}\PYZsh{}\PYZsh{}\PYZsh{}\PYZsh{}\PYZsh{}\PYZsh{}\PYZsh{}\PYZsh{}\PYZsh{}\PYZsh{}\PYZsh{}}
\PY{c+c1}{\PYZsh{}\PYZsh{}\PYZsh{}\PYZsh{}\PYZsh{}\PYZsh{}\PYZsh{}\PYZsh{}\PYZsh{}\PYZsh{}\PYZsh{}\PYZsh{}\PYZsh{}\PYZsh{}\PYZsh{}\PYZsh{}\PYZsh{}\PYZsh{}\PYZsh{}\PYZsh{}\PYZsh{}\PYZsh{}\PYZsh{}\PYZsh{}\PYZsh{}\PYZsh{}\PYZsh{}\PYZsh{}\PYZsh{}\PYZsh{}\PYZsh{}\PYZsh{}\PYZsh{}\PYZsh{}\PYZsh{}\PYZsh{}\PYZsh{}\PYZsh{}\PYZsh{}\PYZsh{}\PYZsh{}\PYZsh{}\PYZsh{}\PYZsh{}\PYZsh{}\PYZsh{}\PYZsh{}\PYZsh{}\PYZsh{}\PYZsh{}\PYZsh{}\PYZsh{}\PYZsh{}\PYZsh{}}

\PY{k}{def} \PY{n+nf}{get\PYZus{}x\PYZus{}distribution\PYZus{}hhl}\PY{p}{(}\PY{n}{shots}\PY{p}{,} \PY{n}{failed\PYZus{}count}\PY{p}{,} \PY{n}{trimmed\PYZus{}counts}\PY{p}{,} \PY{n}{b}\PY{p}{)}\PY{p}{:}
    \PY{c+c1}{\PYZsh{} find probabilities of the top n results (for an n\PYZhy{}dim system)    }
    \PY{n}{top\PYZus{}n\PYZus{}counts} \PY{o}{=} \PY{n+nb}{dict}\PY{p}{(}\PY{n+nb}{sorted}\PY{p}{(}\PY{n}{trimmed\PYZus{}counts}\PY{o}{.}\PY{n}{items}\PY{p}{(}\PY{p}{)}\PY{p}{,} \PY{n}{key}\PY{o}{=}\PY{k}{lambda} \PY{n}{x}\PY{p}{:} \PY{n}{x}\PY{p}{[}\PY{l+m+mi}{1}\PY{p}{]}\PY{p}{,} \PY{n}{reverse}\PY{o}{=}\PY{k+kc}{True}\PY{p}{)}\PY{p}{[}\PY{p}{:}\PY{n+nb}{len}\PY{p}{(}\PY{n}{b}\PY{p}{)}\PY{p}{]}\PY{p}{)}
    \PY{n}{x\PYZus{}hhl} \PY{o}{=} \PY{p}{[}\PY{n}{i}\PY{p}{[}\PY{l+m+mi}{1}\PY{p}{]}\PY{o}{/}\PY{p}{(}\PY{n}{shots} \PY{o}{\PYZhy{}} \PY{n}{failed\PYZus{}count}\PY{p}{)} \PY{k}{for} \PY{n}{i} \PY{o+ow}{in} \PY{n+nb}{sorted}\PY{p}{(}\PY{n}{top\PYZus{}n\PYZus{}counts}\PY{o}{.}\PY{n}{items}\PY{p}{(}\PY{p}{)}\PY{p}{,} \PY{n}{key}\PY{o}{=}\PY{k}{lambda} \PY{n}{i}\PY{p}{:} \PY{n}{i}\PY{p}{[}\PY{l+m+mi}{0}\PY{p}{]}\PY{p}{,} \PY{n}{reverse}\PY{o}{=}\PY{k+kc}{False}\PY{p}{)}\PY{p}{]}
    
    \PY{n}{x\PYZus{}hhl} \PY{o}{=} \PY{n}{x\PYZus{}hhl}
    \PY{k}{return} \PY{n}{np}\PY{o}{.}\PY{n}{array}\PY{p}{(}\PY{n}{x\PYZus{}hhl}\PY{p}{)}

\PY{k}{def} \PY{n+nf}{get\PYZus{}x\PYZus{}distribution\PYZus{}actual}\PY{p}{(}\PY{n}{A}\PY{p}{,} \PY{n}{b}\PY{p}{)}\PY{p}{:}
    \PY{n}{x} \PY{o}{=} \PY{n}{np}\PY{o}{.}\PY{n}{linalg}\PY{o}{.}\PY{n}{solve}\PY{p}{(}\PY{n}{A}\PY{p}{,} \PY{n}{b}\PY{p}{)}
    \PY{c+c1}{\PYZsh{} squaring norm for probability distribution}
    \PY{n}{x} \PY{o}{=} \PY{p}{(}\PY{n}{x}\PY{o}{/}\PY{n}{np}\PY{o}{.}\PY{n}{linalg}\PY{o}{.}\PY{n}{norm}\PY{p}{(}\PY{n}{x}\PY{p}{)}\PY{p}{)}\PY{o}{*}\PY{o}{*}\PY{l+m+mi}{2}
    \PY{k}{return} \PY{n}{np}\PY{o}{.}\PY{n}{array}\PY{p}{(}\PY{n}{x}\PY{p}{)}

\PY{k}{def} \PY{n+nf}{remove\PYZus{}fails}\PY{p}{(}\PY{n}{counts}\PY{p}{)}\PY{p}{:}
    \PY{c+c1}{\PYZsh{} remove all failed runs: either ancilla is |0\PYZgt{} or our QPE register is not all |0\PYZgt{}}
    \PY{n}{failed\PYZus{}count} \PY{o}{=} \PY{n+nb}{sum}\PY{p}{(}\PY{n}{value} \PY{k}{for} \PY{n}{key}\PY{p}{,} \PY{n}{value} \PY{o+ow}{in} \PY{n}{counts}\PY{o}{.}\PY{n}{items}\PY{p}{(}\PY{p}{)} \PY{k}{if} \PY{p}{(}\PY{n}{key}\PY{o}{.}\PY{n}{endswith}\PY{p}{(}\PY{l+s+s2}{\PYZdq{}}\PY{l+s+s2}{0}\PY{l+s+s2}{\PYZdq{}}\PY{p}{)} \PY{o+ow}{or} \PY{n}{key}\PY{p}{[}\PY{o}{\PYZhy{}}\PY{n}{nq}\PY{o}{\PYZhy{}}\PY{l+m+mi}{1}\PY{p}{:}\PY{o}{\PYZhy{}}\PY{l+m+mi}{1}\PY{p}{]} \PY{o}{!=} \PY{p}{(}\PY{l+s+s1}{\PYZsq{}}\PY{l+s+s1}{0}\PY{l+s+s1}{\PYZsq{}}\PY{o}{*}\PY{n}{nq}\PY{p}{)}\PY{p}{)}\PY{p}{)}
    \PY{n}{fails} \PY{o}{=} \PY{p}{[}\PY{p}{]}
    \PY{k}{for} \PY{n}{key}\PY{p}{,} \PY{n}{val} \PY{o+ow}{in} \PY{n}{counts}\PY{o}{.}\PY{n}{items}\PY{p}{(}\PY{p}{)}\PY{p}{:}
        \PY{k}{if} \PY{n}{key}\PY{o}{.}\PY{n}{endswith}\PY{p}{(}\PY{l+s+s2}{\PYZdq{}}\PY{l+s+s2}{0}\PY{l+s+s2}{\PYZdq{}}\PY{p}{)} \PY{o+ow}{or} \PY{n}{key}\PY{p}{[}\PY{o}{\PYZhy{}}\PY{n}{nq}\PY{o}{\PYZhy{}}\PY{l+m+mi}{1}\PY{p}{:}\PY{o}{\PYZhy{}}\PY{l+m+mi}{1}\PY{p}{]} \PY{o}{!=} \PY{p}{(}\PY{l+s+s1}{\PYZsq{}}\PY{l+s+s1}{0}\PY{l+s+s1}{\PYZsq{}}\PY{o}{*}\PY{n}{nq}\PY{p}{)}\PY{p}{:}
            \PY{n}{fails}\PY{o}{.}\PY{n}{append}\PY{p}{(}\PY{n}{key}\PY{p}{)}
    \PY{k}{for} \PY{n}{item} \PY{o+ow}{in} \PY{n}{fails}\PY{p}{:}
        \PY{n}{counts}\PY{o}{.}\PY{n}{pop}\PY{p}{(}\PY{n}{item}\PY{p}{)}
    \PY{k}{return} \PY{n}{failed\PYZus{}count}\PY{p}{,} \PY{n}{counts}
\end{Verbatim}
\end{tcolorbox}

    The following section contains the creation for all major circuit
components of HHL including: initialization, QPE, and Eigenvalue
Rotation. Note: QFT (quantum fourier transform) is implemented from
scratch within QPE (inverse).

    \begin{tcolorbox}[breakable, size=fbox, boxrule=1pt, pad at break*=1mm,colback=cellbackground, colframe=cellborder]
\prompt{In}{incolor}{4}{\boxspacing}
\begin{Verbatim}[commandchars=\\\{\}]
\PY{c+c1}{\PYZsh{}\PYZsh{}\PYZsh{}\PYZsh{}\PYZsh{}\PYZsh{}\PYZsh{}\PYZsh{}\PYZsh{}\PYZsh{}\PYZsh{}\PYZsh{}\PYZsh{}\PYZsh{}\PYZsh{}\PYZsh{}\PYZsh{}\PYZsh{}\PYZsh{}\PYZsh{}\PYZsh{}\PYZsh{}\PYZsh{}\PYZsh{}\PYZsh{}\PYZsh{}\PYZsh{}\PYZsh{}\PYZsh{}\PYZsh{}\PYZsh{}\PYZsh{}\PYZsh{}\PYZsh{}\PYZsh{}\PYZsh{}\PYZsh{}\PYZsh{}\PYZsh{}\PYZsh{}\PYZsh{}\PYZsh{}\PYZsh{}\PYZsh{}\PYZsh{}\PYZsh{}\PYZsh{}\PYZsh{}\PYZsh{}\PYZsh{}\PYZsh{}\PYZsh{}\PYZsh{}\PYZsh{}}
\PY{c+c1}{\PYZsh{}\PYZsh{}\PYZsh{}\PYZsh{}\PYZsh{}\PYZsh{}\PYZsh{}\PYZsh{}\PYZsh{}\PYZsh{}\PYZsh{}\PYZsh{}\PYZsh{}\PYZsh{}\PYZsh{}\PYZsh{} Circuit Components \PYZsh{}\PYZsh{}\PYZsh{}\PYZsh{}\PYZsh{}\PYZsh{}\PYZsh{}\PYZsh{}\PYZsh{}\PYZsh{}\PYZsh{}\PYZsh{}\PYZsh{}\PYZsh{}\PYZsh{}\PYZsh{}\PYZsh{}\PYZsh{}}
\PY{c+c1}{\PYZsh{}\PYZsh{}\PYZsh{}\PYZsh{}\PYZsh{}\PYZsh{}\PYZsh{}\PYZsh{}\PYZsh{}\PYZsh{}\PYZsh{}\PYZsh{}\PYZsh{}\PYZsh{}\PYZsh{}\PYZsh{}\PYZsh{}\PYZsh{}\PYZsh{}\PYZsh{}\PYZsh{}\PYZsh{}\PYZsh{}\PYZsh{}\PYZsh{}\PYZsh{}\PYZsh{}\PYZsh{}\PYZsh{}\PYZsh{}\PYZsh{}\PYZsh{}\PYZsh{}\PYZsh{}\PYZsh{}\PYZsh{}\PYZsh{}\PYZsh{}\PYZsh{}\PYZsh{}\PYZsh{}\PYZsh{}\PYZsh{}\PYZsh{}\PYZsh{}\PYZsh{}\PYZsh{}\PYZsh{}\PYZsh{}\PYZsh{}\PYZsh{}\PYZsh{}\PYZsh{}\PYZsh{}}

\PY{c+c1}{\PYZsh{}\PYZsh{}\PYZsh{}\PYZsh{}\PYZsh{}\PYZsh{}\PYZsh{}\PYZsh{}\PYZsh{}\PYZsh{}\PYZsh{} Initialization b Circuit \PYZsh{}\PYZsh{}\PYZsh{}\PYZsh{}\PYZsh{}\PYZsh{}\PYZsh{}\PYZsh{}\PYZsh{}\PYZsh{}\PYZsh{}}
\PY{k}{def} \PY{n+nf}{initb\PYZus{}construct\PYZus{}circuit}\PY{p}{(}
        \PY{n}{n\PYZus{}b}\PY{p}{:} \PY{n}{QuantumRegister}\PY{p}{,}
        \PY{n}{n\PYZus{}q}\PY{p}{:} \PY{n}{QuantumRegister}\PY{p}{,}
        \PY{n}{n\PYZus{}a}\PY{p}{:} \PY{n}{QuantumRegister}\PY{p}{,}
        \PY{n}{cb}\PY{p}{:} \PY{n}{ClassicalRegister}\PY{p}{,}
        \PY{n}{b}\PY{p}{:} \PY{n}{np}\PY{o}{.}\PY{n}{ndarray}\PY{p}{,}
\PY{p}{)} \PY{o}{\PYZhy{}}\PY{o}{\PYZgt{}} \PY{n}{QuantumCircuit}\PY{p}{:}
    \PY{n}{init\PYZus{}circ} \PY{o}{=} \PY{n}{QuantumCircuit}\PY{p}{(}\PY{n}{n\PYZus{}b}\PY{p}{,} \PY{n}{n\PYZus{}q}\PY{p}{,} \PY{n}{n\PYZus{}a}\PY{p}{,} \PY{n}{cb}\PY{p}{)}
    \PY{n}{init\PYZus{}circ}\PY{o}{.}\PY{n}{initialize}\PY{p}{(}\PY{n}{b}\PY{o}{/}\PY{n}{np}\PY{o}{.}\PY{n}{linalg}\PY{o}{.}\PY{n}{norm}\PY{p}{(}\PY{n}{b}\PY{p}{)}\PY{p}{,} \PY{n}{n\PYZus{}b}\PY{p}{)}
    \PY{k}{return} \PY{n}{init\PYZus{}circ}

\PY{c+c1}{\PYZsh{}\PYZsh{}\PYZsh{}\PYZsh{}\PYZsh{}\PYZsh{}\PYZsh{}\PYZsh{}\PYZsh{}\PYZsh{}\PYZsh{}\PYZsh{}\PYZsh{}\PYZsh{}\PYZsh{}\PYZsh{} QPE Circuit \PYZsh{}\PYZsh{}\PYZsh{}\PYZsh{}\PYZsh{}\PYZsh{}\PYZsh{}\PYZsh{}\PYZsh{}\PYZsh{}\PYZsh{}\PYZsh{}\PYZsh{}\PYZsh{}\PYZsh{}\PYZsh{}}
\PY{k}{def} \PY{n+nf}{QPE\PYZus{}construct\PYZus{}circuit}\PY{p}{(}
        \PY{n}{n\PYZus{}b}\PY{p}{:} \PY{n}{QuantumRegister}\PY{p}{,}
        \PY{n}{n\PYZus{}q}\PY{p}{:} \PY{n}{QuantumRegister}\PY{p}{,}
        \PY{n}{n\PYZus{}a}\PY{p}{:} \PY{n}{QuantumRegister}\PY{p}{,}
        \PY{n}{cb}\PY{p}{:} \PY{n}{ClassicalRegister}\PY{p}{,}
        \PY{n}{U}\PY{p}{:} \PY{n}{np}\PY{o}{.}\PY{n}{ndarray}\PY{p}{,}
\PY{p}{)} \PY{o}{\PYZhy{}}\PY{o}{\PYZgt{}} \PY{n}{QuantumCircuit}\PY{p}{:}
    \PY{n}{nq} \PY{o}{=} \PY{n}{n\PYZus{}q}\PY{o}{.}\PY{n}{size}
    \PY{n}{nb} \PY{o}{=} \PY{n}{n\PYZus{}b}\PY{o}{.}\PY{n}{size}
    \PY{n}{qpe\PYZus{}circ} \PY{o}{=} \PY{n}{QuantumCircuit}\PY{p}{(}\PY{n}{n\PYZus{}b}\PY{p}{,} \PY{n}{n\PYZus{}q}\PY{p}{,} \PY{n}{n\PYZus{}a}\PY{p}{,} \PY{n}{cb}\PY{p}{)}
    \PY{c+c1}{\PYZsh{} Hadamard all QPE register qubits}
    \PY{n}{qpe\PYZus{}circ}\PY{o}{.}\PY{n}{h}\PY{p}{(}\PY{n}{n\PYZus{}q}\PY{p}{)}
    
    \PY{c+c1}{\PYZsh{} Controlled U Operations}
    \PY{n}{U} \PY{o}{=} \PY{n}{UnitaryGate}\PY{p}{(}\PY{n}{U}\PY{p}{,} \PY{n}{label}\PY{o}{=}\PY{l+s+s1}{\PYZsq{}}\PY{l+s+s1}{U}\PY{l+s+s1}{\PYZsq{}}\PY{p}{)}
    \PY{k}{for} \PY{n}{i} \PY{o+ow}{in} \PY{n+nb}{range}\PY{p}{(}\PY{n}{nq}\PY{p}{)}\PY{p}{:}
        \PY{n}{U\PYZus{}k} \PY{o}{=} \PY{n}{U}\PY{o}{.}\PY{n}{power}\PY{p}{(}\PY{l+m+mi}{2}\PY{o}{*}\PY{o}{*}\PY{n}{i}\PY{p}{)} 
        \PY{n}{ctrl\PYZus{}U\PYZus{}k} \PY{o}{=} \PY{n}{U\PYZus{}k}\PY{o}{.}\PY{n}{control}\PY{p}{(}\PY{p}{)}
        \PY{n}{qpe\PYZus{}circ}\PY{o}{.}\PY{n}{append}\PY{p}{(}\PY{n}{ctrl\PYZus{}U\PYZus{}k}\PY{p}{,} \PY{n+nb}{list}\PY{p}{(}\PY{n}{np}\PY{o}{.}\PY{n}{concatenate}\PY{p}{(}\PY{p}{(}\PY{p}{[}\PY{n}{nb}\PY{o}{+}\PY{n}{i}\PY{p}{]}\PY{p}{,} \PY{n}{np}\PY{o}{.}\PY{n}{arange}\PY{p}{(}\PY{n}{nb}\PY{p}{)}\PY{p}{)}\PY{p}{)}\PY{p}{)}\PY{p}{)}
    \PY{n}{qpe\PYZus{}circ}\PY{o}{.}\PY{n}{barrier}\PY{p}{(}\PY{p}{)}
    
    \PY{c+c1}{\PYZsh{} Inverse QFT}
    \PY{k}{for} \PY{n}{qubit} \PY{o+ow}{in} \PY{n+nb}{range}\PY{p}{(}\PY{n}{nq}\PY{o}{/}\PY{o}{/}\PY{l+m+mi}{2}\PY{p}{)}\PY{p}{:}
        \PY{n}{qpe\PYZus{}circ}\PY{o}{.}\PY{n}{swap}\PY{p}{(}\PY{n}{n\PYZus{}q}\PY{p}{[}\PY{n}{nq}\PY{o}{\PYZhy{}}\PY{n}{qubit}\PY{o}{\PYZhy{}}\PY{l+m+mi}{1}\PY{p}{]}\PY{p}{,} \PY{n}{n\PYZus{}q}\PY{p}{[}\PY{n}{qubit}\PY{p}{]}\PY{p}{)} 
    \PY{k}{for} \PY{n}{i} \PY{o+ow}{in} \PY{n+nb}{range}\PY{p}{(}\PY{n}{nq}\PY{p}{)}\PY{p}{:}
        \PY{k}{for} \PY{n}{m} \PY{o+ow}{in} \PY{n+nb}{range}\PY{p}{(}\PY{n}{i}\PY{p}{)}\PY{p}{:}
            \PY{n}{qpe\PYZus{}circ}\PY{o}{.}\PY{n}{cp}\PY{p}{(}\PY{o}{\PYZhy{}}\PY{n}{pi}\PY{o}{/}\PY{p}{(}\PY{l+m+mi}{2}\PY{o}{*}\PY{o}{*}\PY{p}{(}\PY{n}{i}\PY{o}{\PYZhy{}}\PY{n}{m}\PY{p}{)}\PY{p}{)}\PY{p}{,} \PY{n}{n\PYZus{}q}\PY{p}{[}\PY{n}{m}\PY{p}{]}\PY{p}{,} \PY{n}{n\PYZus{}q}\PY{p}{[}\PY{n}{i}\PY{p}{]}\PY{p}{)}
        \PY{n}{qpe\PYZus{}circ}\PY{o}{.}\PY{n}{h}\PY{p}{(}\PY{n}{n\PYZus{}q}\PY{p}{[}\PY{n}{i}\PY{p}{]}\PY{p}{)}
        
    \PY{k}{return} \PY{n}{qpe\PYZus{}circ}

\PY{c+c1}{\PYZsh{}\PYZsh{}\PYZsh{}\PYZsh{}\PYZsh{}\PYZsh{}\PYZsh{}\PYZsh{}\PYZsh{}\PYZsh{} Eigenvalue Rotation Circuit \PYZsh{}\PYZsh{}\PYZsh{}\PYZsh{}\PYZsh{}\PYZsh{}\PYZsh{}\PYZsh{}\PYZsh{}}
\PY{k}{def} \PY{n+nf}{eigrot\PYZus{}construct\PYZus{}circuit}\PY{p}{(}
        \PY{n}{b}\PY{p}{:} \PY{n}{np}\PY{o}{.}\PY{n}{ndarray}\PY{p}{,}
        \PY{n}{A}\PY{p}{:} \PY{n}{np}\PY{o}{.}\PY{n}{ndarray}\PY{p}{,}
        \PY{n}{U}\PY{p}{:} \PY{n}{np}\PY{o}{.}\PY{n}{ndarray}\PY{p}{,}
        \PY{n}{n\PYZus{}b}\PY{p}{:} \PY{n}{QuantumRegister}\PY{p}{,}
        \PY{n}{n\PYZus{}q}\PY{p}{:} \PY{n}{QuantumRegister}\PY{p}{,}
        \PY{n}{n\PYZus{}a}\PY{p}{:} \PY{n}{QuantumRegister}\PY{p}{,}
        \PY{n}{cb}\PY{p}{:} \PY{n}{ClassicalRegister}\PY{p}{,}
        \PY{n}{t}\PY{p}{:} \PY{n+nb}{float}\PY{p}{,}
        \PY{n}{normal}\PY{p}{:} \PY{n+nb}{bool}\PY{p}{,} \PY{c+c1}{\PYZsh{} see retrieve\PYZus{}angles function}
\PY{p}{)} \PY{o}{\PYZhy{}}\PY{o}{\PYZgt{}} \PY{n}{QuantumCircuit}\PY{p}{:}
    \PY{n}{nq} \PY{o}{=} \PY{n}{n\PYZus{}q}\PY{o}{.}\PY{n}{size}
    \PY{n}{nb} \PY{o}{=} \PY{n}{n\PYZus{}b}\PY{o}{.}\PY{n}{size}
    \PY{c+c1}{\PYZsh{} retrieve the lambda\PYZus{}eigenvalues and the necessary angles of rotation}
    \PY{n}{eigenvalues\PYZus{}qpe}\PY{p}{,} \PY{n}{eigenvalue\PYZus{}oracle} \PY{o}{=} \PY{n}{QPE\PYZus{}extractor}\PY{p}{(}\PY{n}{b}\PY{p}{,}\PY{n}{A}\PY{p}{,}\PY{n}{U}\PY{p}{,}\PY{n}{nb}\PY{p}{,}\PY{n}{nq}\PY{p}{,}\PY{n}{t}\PY{p}{)}
    \PY{n}{thetas} \PY{o}{=} \PY{n}{retrieve\PYZus{}angles}\PY{p}{(}\PY{n}{eigenvalues\PYZus{}qpe}\PY{p}{,} \PY{n}{eigenvalue\PYZus{}oracle}\PY{p}{,} \PY{n}{t}\PY{p}{,} \PY{n}{normal}\PY{p}{)}
    \PY{n}{ev\PYZus{}bitstrings} \PY{o}{=} \PY{n+nb}{list}\PY{p}{(}\PY{n}{eigenvalues\PYZus{}qpe}\PY{o}{.}\PY{n}{values}\PY{p}{(}\PY{p}{)}\PY{p}{)}
    
    \PY{n}{eigrot\PYZus{}circ} \PY{o}{=} \PY{n}{QuantumCircuit}\PY{p}{(}\PY{n}{n\PYZus{}b}\PY{p}{,} \PY{n}{n\PYZus{}q}\PY{p}{,} \PY{n}{n\PYZus{}a}\PY{p}{,} \PY{n}{cb}\PY{p}{)}
    \PY{c+c1}{\PYZsh{} constructing the multi\PYZhy{}controlled rotation gates}
    \PY{k}{for} \PY{n}{ind}\PY{p}{,} \PY{n}{theta} \PY{o+ow}{in} \PY{n+nb}{enumerate}\PY{p}{(}\PY{n}{thetas}\PY{p}{)}\PY{p}{:}
        \PY{c+c1}{\PYZsh{} the control state simply corresponds to the bit representation of the eigenvalue}
        \PY{n}{CU\PYZus{}gate} \PY{o}{=} \PY{n}{UGate}\PY{p}{(}\PY{n}{theta}\PY{p}{,} \PY{l+m+mi}{0}\PY{p}{,} \PY{l+m+mi}{0}\PY{p}{)}\PY{o}{.}\PY{n}{control}\PY{p}{(}\PY{n}{nq}\PY{p}{,} \PY{n}{ctrl\PYZus{}state} \PY{o}{=} \PY{n}{ev\PYZus{}bitstrings}\PY{p}{[}\PY{n}{ind}\PY{p}{]}\PY{p}{)}                
        \PY{n}{eigrot\PYZus{}circ}\PY{o}{.}\PY{n}{append}\PY{p}{(}\PY{n}{CU\PYZus{}gate}\PY{p}{,} \PY{n+nb}{list}\PY{p}{(}\PY{n}{np}\PY{o}{.}\PY{n}{concatenate}\PY{p}{(}\PY{p}{(}\PY{n}{np}\PY{o}{.}\PY{n}{arange}\PY{p}{(}\PY{n}{nb}\PY{p}{,}\PY{n}{nq}\PY{o}{+}\PY{n}{nb}\PY{p}{)}\PY{p}{,} \PY{p}{[}\PY{n}{nb}\PY{o}{+}\PY{n}{nq}\PY{p}{]}\PY{p}{)}\PY{p}{)}\PY{p}{)}\PY{p}{)}
    
    \PY{k}{return} \PY{n}{eigrot\PYZus{}circ}
\end{Verbatim}
\end{tcolorbox}

    One of the weaknesses of HHL is the fact that it requires an ``oracle''
of sorts that knows the actual eigenvalues and can compare the
approximations of QPE to these expected eigenvalues in order to
determine their parity and thus the rotation angles needed to flip them.
Many implementations that exist require manual changes and calculations
- our implementation removes this step using quantum circuit composition
(a quantum subroutine). Specifically, we perform QPE, extract the
respective lambda eigenvalues, and compare them to classically computed
eigenvalues. This is what makes the algorithm hybrid.

    \begin{tcolorbox}[breakable, size=fbox, boxrule=1pt, pad at break*=1mm,colback=cellbackground, colframe=cellborder]
\prompt{In}{incolor}{5}{\boxspacing}
\begin{Verbatim}[commandchars=\\\{\}]
\PY{c+c1}{\PYZsh{}\PYZsh{}\PYZsh{}\PYZsh{}\PYZsh{}\PYZsh{}\PYZsh{}\PYZsh{}\PYZsh{}\PYZsh{}\PYZsh{}\PYZsh{}\PYZsh{}\PYZsh{}\PYZsh{}\PYZsh{}\PYZsh{}\PYZsh{}\PYZsh{}\PYZsh{}\PYZsh{}\PYZsh{}\PYZsh{}\PYZsh{}\PYZsh{}\PYZsh{}\PYZsh{}\PYZsh{}\PYZsh{}\PYZsh{}\PYZsh{}\PYZsh{}\PYZsh{}\PYZsh{}\PYZsh{}\PYZsh{}\PYZsh{}\PYZsh{}\PYZsh{}\PYZsh{}\PYZsh{}\PYZsh{}\PYZsh{}\PYZsh{}\PYZsh{}\PYZsh{}\PYZsh{}\PYZsh{}\PYZsh{}\PYZsh{}\PYZsh{}\PYZsh{}\PYZsh{}\PYZsh{}}
\PY{c+c1}{\PYZsh{}\PYZsh{}\PYZsh{}\PYZsh{}\PYZsh{}\PYZsh{}\PYZsh{}\PYZsh{}\PYZsh{}\PYZsh{}\PYZsh{}\PYZsh{} QPE Extractor and Verifiers \PYZsh{}\PYZsh{}\PYZsh{}\PYZsh{}\PYZsh{}\PYZsh{}\PYZsh{}\PYZsh{}\PYZsh{}\PYZsh{}\PYZsh{}\PYZsh{}\PYZsh{}}
\PY{c+c1}{\PYZsh{}\PYZsh{}\PYZsh{}\PYZsh{}\PYZsh{}\PYZsh{}\PYZsh{}\PYZsh{}\PYZsh{}\PYZsh{}\PYZsh{}\PYZsh{}\PYZsh{}\PYZsh{}\PYZsh{}\PYZsh{}\PYZsh{}\PYZsh{}\PYZsh{}\PYZsh{}\PYZsh{}\PYZsh{}\PYZsh{}\PYZsh{}\PYZsh{}\PYZsh{}\PYZsh{}\PYZsh{}\PYZsh{}\PYZsh{}\PYZsh{}\PYZsh{}\PYZsh{}\PYZsh{}\PYZsh{}\PYZsh{}\PYZsh{}\PYZsh{}\PYZsh{}\PYZsh{}\PYZsh{}\PYZsh{}\PYZsh{}\PYZsh{}\PYZsh{}\PYZsh{}\PYZsh{}\PYZsh{}\PYZsh{}\PYZsh{}\PYZsh{}\PYZsh{}\PYZsh{}\PYZsh{}}
\PY{k}{def} \PY{n+nf}{QPE\PYZus{}extractor\PYZus{}circuit}\PY{p}{(}
        \PY{n}{b}\PY{p}{:} \PY{n}{np}\PY{o}{.}\PY{n}{ndarray}\PY{p}{,}
        \PY{n}{U}\PY{p}{:} \PY{n}{np}\PY{o}{.}\PY{n}{ndarray}\PY{p}{,}
        \PY{n}{nb}\PY{p}{:} \PY{n+nb}{int}\PY{p}{,}
        \PY{n}{nq}\PY{p}{:} \PY{n+nb}{int}\PY{p}{,}
\PY{p}{)}\PY{o}{\PYZhy{}}\PY{o}{\PYZgt{}} \PY{n}{QuantumCircuit}\PY{p}{:}
    \PY{n}{n\PYZus{}b} \PY{o}{=} \PY{n}{QuantumRegister}\PY{p}{(}\PY{n}{nb}\PY{p}{,} \PY{l+s+s1}{\PYZsq{}}\PY{l+s+s1}{b}\PY{l+s+s1}{\PYZsq{}}\PY{p}{)}
    \PY{n}{n\PYZus{}q} \PY{o}{=} \PY{n}{QuantumRegister}\PY{p}{(}\PY{n}{nq}\PY{p}{,} \PY{l+s+s1}{\PYZsq{}}\PY{l+s+s1}{QPE}\PY{l+s+s1}{\PYZsq{}}\PY{p}{)}
    \PY{n}{n\PYZus{}a} \PY{o}{=} \PY{n}{QuantumRegister}\PY{p}{(}\PY{l+m+mi}{1}\PY{p}{,} \PY{l+s+s1}{\PYZsq{}}\PY{l+s+s1}{aux}\PY{l+s+s1}{\PYZsq{}}\PY{p}{)}
    \PY{n}{cb} \PY{o}{=} \PY{n}{ClassicalRegister}\PY{p}{(}\PY{n}{nq}\PY{p}{,} \PY{l+s+s1}{\PYZsq{}}\PY{l+s+s1}{cb}\PY{l+s+s1}{\PYZsq{}}\PY{p}{)}
    
    \PY{c+c1}{\PYZsh{}\PYZsh{}\PYZsh{}\PYZsh{} INITIALIZE B \PYZsh{}\PYZsh{}\PYZsh{}\PYZsh{}}
    \PY{n}{init\PYZus{}circ} \PY{o}{=} \PY{n}{initb\PYZus{}construct\PYZus{}circuit}\PY{p}{(}\PY{n}{n\PYZus{}b}\PY{p}{,} \PY{n}{n\PYZus{}q}\PY{p}{,} \PY{n}{n\PYZus{}a}\PY{p}{,} \PY{n}{cb}\PY{p}{,} \PY{n}{b}\PY{p}{)}
    \PY{n}{init\PYZus{}circ}\PY{o}{.}\PY{n}{barrier}\PY{p}{(}\PY{p}{)}
    
    \PY{c+c1}{\PYZsh{}\PYZsh{}\PYZsh{}\PYZsh{} QPE CIRCUIT \PYZsh{}\PYZsh{}\PYZsh{}\PYZsh{}}
    \PY{n}{qpe\PYZus{}circ} \PY{o}{=} \PY{n}{QPE\PYZus{}construct\PYZus{}circuit}\PY{p}{(}\PY{n}{n\PYZus{}b}\PY{p}{,} \PY{n}{n\PYZus{}q}\PY{p}{,} \PY{n}{n\PYZus{}a}\PY{p}{,} \PY{n}{cb}\PY{p}{,} \PY{n}{U}\PY{p}{)}
    \PY{n}{qpe\PYZus{}circ}\PY{o}{.}\PY{n}{barrier}\PY{p}{(}\PY{p}{)}
    
    \PY{c+c1}{\PYZsh{}\PYZsh{}\PYZsh{}\PYZsh{} MEASURE QPE \PYZsh{}\PYZsh{}\PYZsh{}\PYZsh{}}
    \PY{n}{measureQPE\PYZus{}circ} \PY{o}{=} \PY{n}{QuantumCircuit}\PY{p}{(}\PY{n}{n\PYZus{}b}\PY{p}{,} \PY{n}{n\PYZus{}q}\PY{p}{,} \PY{n}{n\PYZus{}a}\PY{p}{,} \PY{n}{cb}\PY{p}{)}
    \PY{n}{measureQPE\PYZus{}circ}\PY{o}{.}\PY{n}{measure}\PY{p}{(}\PY{n}{n\PYZus{}q}\PY{p}{,} \PY{n}{cb}\PY{p}{)}
    \PY{n}{measureQPE\PYZus{}circ}\PY{o}{.}\PY{n}{barrier}\PY{p}{(}\PY{p}{)}
    
    \PY{n}{qpe\PYZus{}circuit} \PY{o}{=} \PY{n}{init\PYZus{}circ}\PY{o}{.}\PY{n}{compose}\PY{p}{(}\PY{n}{qpe\PYZus{}circ}\PY{p}{)}
    \PY{n}{qpe\PYZus{}circuit} \PY{o}{=} \PY{n}{qpe\PYZus{}circuit}\PY{o}{.}\PY{n}{compose}\PY{p}{(}\PY{n}{measureQPE\PYZus{}circ}\PY{p}{)}
    \PY{k}{return} \PY{n}{qpe\PYZus{}circuit}

\PY{k}{def} \PY{n+nf}{QPE\PYZus{}extractor}\PY{p}{(}
        \PY{n}{b}\PY{p}{:} \PY{n}{np}\PY{o}{.}\PY{n}{ndarray}\PY{p}{,}
        \PY{n}{A}\PY{p}{:} \PY{n}{np}\PY{o}{.}\PY{n}{ndarray}\PY{p}{,}
        \PY{n}{U}\PY{p}{:} \PY{n}{np}\PY{o}{.}\PY{n}{ndarray}\PY{p}{,}
        \PY{n}{nb}\PY{p}{:} \PY{n+nb}{int}\PY{p}{,}
        \PY{n}{nq}\PY{p}{:} \PY{n+nb}{int}\PY{p}{,}
        \PY{n}{t}\PY{p}{:} \PY{n+nb}{float}\PY{p}{,}
\PY{p}{)} \PY{o}{\PYZhy{}}\PY{o}{\PYZgt{}} \PY{p}{(}\PY{n+nb}{dict}\PY{p}{,} \PY{n+nb}{dict}\PY{p}{)}\PY{p}{:}
    \PY{n}{circuit} \PY{o}{=} \PY{n}{QPE\PYZus{}extractor\PYZus{}circuit}\PY{p}{(}\PY{n}{b}\PY{p}{,} \PY{n}{U}\PY{p}{,} \PY{n}{nb}\PY{p}{,} \PY{n}{nq}\PY{p}{)}
    \PY{n}{job} \PY{o}{=} \PY{n}{execute}\PY{p}{(}\PY{n}{circuit}\PY{p}{,} \PY{n}{backend}\PY{o}{=}\PY{n}{backend\PYZus{}qasm}\PY{p}{,} \PY{n}{shots}\PY{o}{=}\PY{l+m+mi}{1000}\PY{p}{)} \PY{c+c1}{\PYZsh{} execute the QPE circuit}
    \PY{n}{counts} \PY{o}{=} \PY{n}{job}\PY{o}{.}\PY{n}{result}\PY{p}{(}\PY{p}{)}\PY{o}{.}\PY{n}{get\PYZus{}counts}\PY{p}{(}\PY{p}{)}
    
    \PY{c+c1}{\PYZsh{} retrieve the top bitstrings corresponding to the lambda\PYZus{}eigenvalues}
    \PY{n}{top\PYZus{}bitstrings} \PY{o}{=} \PY{p}{[}\PY{n}{i}\PY{p}{[}\PY{l+m+mi}{0}\PY{p}{]} \PY{k}{for} \PY{n}{i} \PY{o+ow}{in} \PY{n+nb}{sorted}\PY{p}{(}\PY{n}{counts}\PY{o}{.}\PY{n}{items}\PY{p}{(}\PY{p}{)}\PY{p}{,} \PY{n}{key}\PY{o}{=}\PY{k}{lambda} \PY{n}{i}\PY{p}{:} \PY{n}{i}\PY{p}{[}\PY{l+m+mi}{1}\PY{p}{]}\PY{p}{,} \PY{n}{reverse}\PY{o}{=}\PY{k+kc}{True}\PY{p}{)}\PY{p}{[}\PY{p}{:}\PY{n+nb}{len}\PY{p}{(}\PY{n}{b}\PY{p}{)}\PY{p}{]}\PY{p}{]}
    \PY{n}{dec\PYZus{}lambda\PYZus{}eigenvalue} \PY{o}{=} \PY{n}{np}\PY{o}{.}\PY{n}{array}\PY{p}{(}\PY{p}{[}\PY{n+nb}{int}\PY{p}{(}\PY{n}{i}\PY{p}{[}\PY{p}{:}\PY{p}{:}\PY{l+m+mi}{1}\PY{p}{]}\PY{p}{,}\PY{l+m+mi}{2}\PY{p}{)} \PY{k}{for} \PY{n}{i} \PY{o+ow}{in} \PY{n}{top\PYZus{}bitstrings}\PY{p}{]}\PY{p}{)}
    \PY{n}{lambda\PYZus{}eigenvalues} \PY{o}{=} \PY{n+nb}{dict}\PY{p}{(}\PY{n+nb}{zip}\PY{p}{(}\PY{n}{dec\PYZus{}lambda\PYZus{}eigenvalue}\PY{p}{,} \PY{n}{top\PYZus{}bitstrings}\PY{p}{)}\PY{p}{)}
    
    \PY{c+c1}{\PYZsh{} calculate classically the eigenvalues in order to determine each eigenvalues\PYZsq{} parity}
    \PY{n}{eig\PYZus{}val}\PY{p}{,} \PY{n}{eig\PYZus{}vec} \PY{o}{=} \PY{n}{np}\PY{o}{.}\PY{n}{linalg}\PY{o}{.}\PY{n}{eigh}\PY{p}{(}\PY{n}{A}\PY{p}{)}
    \PY{n}{exp\PYZus{}lambda\PYZus{}eigenvalues} \PY{o}{=} \PY{n}{expected\PYZus{}lambda\PYZus{}eigenvalues}\PY{p}{(}\PY{n}{np}\PY{o}{.}\PY{n}{real}\PY{p}{(}\PY{n}{eig\PYZus{}val}\PY{p}{)}\PY{p}{,} \PY{n}{nq}\PY{p}{,} \PY{n}{t}\PY{p}{)}
    \PY{n}{eigenvalue\PYZus{}oracle} \PY{o}{=} \PY{n+nb}{dict}\PY{p}{(}\PY{n+nb}{zip}\PY{p}{(}\PY{n}{exp\PYZus{}lambda\PYZus{}eigenvalues}\PY{p}{,} \PY{n}{np}\PY{o}{.}\PY{n}{real}\PY{p}{(}\PY{n}{eig\PYZus{}val}\PY{p}{)}\PY{p}{)}\PY{p}{)}
    
    \PY{c+c1}{\PYZsh{} pass information as zipped dictionaries}
    \PY{k}{return} \PY{n}{lambda\PYZus{}eigenvalues}\PY{p}{,} \PY{n}{eigenvalue\PYZus{}oracle}

\PY{k}{def} \PY{n+nf}{expected\PYZus{}lambda\PYZus{}eigenvalues}\PY{p}{(}
        \PY{n}{eig\PYZus{}val}\PY{p}{:} \PY{n}{np}\PY{o}{.}\PY{n}{ndarray}\PY{p}{,}
        \PY{n}{nq}\PY{p}{:} \PY{n+nb}{int}\PY{p}{,}
        \PY{n}{t}\PY{p}{:} \PY{n+nb}{float}\PY{p}{,}
\PY{p}{)}\PY{p}{:}
    \PY{c+c1}{\PYZsh{} calculating the expected lambda\PYZus{}eigenvalue}
    \PY{k}{return} \PY{p}{[}\PY{n+nb}{int}\PY{p}{(}\PY{l+m+mi}{2}\PY{o}{*}\PY{o}{*}\PY{n}{nq} \PY{o}{*} \PY{n}{t} \PY{o}{*} \PY{n}{val} \PY{o}{/} \PY{p}{(}\PY{l+m+mi}{2}\PY{o}{*}\PY{n}{pi}\PY{p}{)} \PY{o}{\PYZpc{}} \PY{p}{(}\PY{l+m+mi}{2}\PY{o}{*}\PY{o}{*}\PY{n}{nq}\PY{p}{)}\PY{p}{)} \PY{k}{for} \PY{n}{val} \PY{o+ow}{in} \PY{n}{eig\PYZus{}val}\PY{p}{]}
\end{Verbatim}
\end{tcolorbox}

    Below contains helper functions for creating the eigenvalue inverting
circuit. Using the quantum subroutine above, it retrieves the necessary
angles to rotate by, corrected with eigenvalue parity information
automatically.

    \begin{tcolorbox}[breakable, size=fbox, boxrule=1pt, pad at break*=1mm,colback=cellbackground, colframe=cellborder]
\prompt{In}{incolor}{6}{\boxspacing}
\begin{Verbatim}[commandchars=\\\{\}]
\PY{c+c1}{\PYZsh{}\PYZsh{}\PYZsh{}\PYZsh{}\PYZsh{}\PYZsh{}\PYZsh{}\PYZsh{}\PYZsh{}\PYZsh{}\PYZsh{}\PYZsh{}\PYZsh{}\PYZsh{}\PYZsh{}\PYZsh{}\PYZsh{}\PYZsh{}\PYZsh{}\PYZsh{}\PYZsh{}\PYZsh{}\PYZsh{}\PYZsh{}\PYZsh{}\PYZsh{}\PYZsh{}\PYZsh{}\PYZsh{}\PYZsh{}\PYZsh{}\PYZsh{}\PYZsh{}\PYZsh{}\PYZsh{}\PYZsh{}\PYZsh{}\PYZsh{}\PYZsh{}\PYZsh{}\PYZsh{}\PYZsh{}\PYZsh{}\PYZsh{}\PYZsh{}\PYZsh{}\PYZsh{}\PYZsh{}\PYZsh{}\PYZsh{}\PYZsh{}\PYZsh{}\PYZsh{}\PYZsh{}}
\PY{c+c1}{\PYZsh{}\PYZsh{}\PYZsh{}\PYZsh{}\PYZsh{}\PYZsh{}\PYZsh{}\PYZsh{}\PYZsh{}\PYZsh{}\PYZsh{} Eigenvalue Rotator and Angles \PYZsh{}\PYZsh{}\PYZsh{}\PYZsh{}\PYZsh{}\PYZsh{}\PYZsh{}\PYZsh{}\PYZsh{}\PYZsh{}\PYZsh{}\PYZsh{}}
\PY{c+c1}{\PYZsh{}\PYZsh{}\PYZsh{}\PYZsh{}\PYZsh{}\PYZsh{}\PYZsh{}\PYZsh{}\PYZsh{}\PYZsh{}\PYZsh{}\PYZsh{}\PYZsh{}\PYZsh{}\PYZsh{}\PYZsh{}\PYZsh{}\PYZsh{}\PYZsh{}\PYZsh{}\PYZsh{}\PYZsh{}\PYZsh{}\PYZsh{}\PYZsh{}\PYZsh{}\PYZsh{}\PYZsh{}\PYZsh{}\PYZsh{}\PYZsh{}\PYZsh{}\PYZsh{}\PYZsh{}\PYZsh{}\PYZsh{}\PYZsh{}\PYZsh{}\PYZsh{}\PYZsh{}\PYZsh{}\PYZsh{}\PYZsh{}\PYZsh{}\PYZsh{}\PYZsh{}\PYZsh{}\PYZsh{}\PYZsh{}\PYZsh{}\PYZsh{}\PYZsh{}\PYZsh{}\PYZsh{}}
\PY{k}{def} \PY{n+nf}{retrieve\PYZus{}angles}\PY{p}{(}
        \PY{n}{lambda\PYZus{}eigenvalues}\PY{p}{:} \PY{n+nb}{dict}\PY{p}{,}
        \PY{n}{eigenvalue\PYZus{}oracle}\PY{p}{:} \PY{n+nb}{dict}\PY{p}{,}
        \PY{n}{t}\PY{p}{:} \PY{n+nb}{float}\PY{p}{,}
        \PY{n}{normal}\PY{p}{:} \PY{n+nb}{bool}\PY{p}{,} \PY{c+c1}{\PYZsh{} this tells us whether we are performing a normal hhl, or a variant like in section 5}
\PY{p}{)} \PY{o}{\PYZhy{}}\PY{o}{\PYZgt{}} \PY{n}{np}\PY{o}{.}\PY{n}{ndarray}\PY{p}{:}
    \PY{n}{lambda\PYZus{}js} \PY{o}{=} \PY{p}{[}\PY{p}{]}
    \PY{n}{C} \PY{o}{=} \PY{k+kc}{None}
    \PY{k}{for} \PY{n}{lambda\PYZus{}ev} \PY{o+ow}{in} \PY{n}{lambda\PYZus{}eigenvalues}\PY{o}{.}\PY{n}{keys}\PY{p}{(}\PY{p}{)}\PY{p}{:}
        \PY{n}{closest\PYZus{}exp\PYZus{}ev} \PY{o}{=} \PY{n+nb}{min}\PY{p}{(}\PY{n}{eigenvalue\PYZus{}oracle}\PY{o}{.}\PY{n}{keys}\PY{p}{(}\PY{p}{)}\PY{p}{,} \PY{n}{key}\PY{o}{=}\PY{k}{lambda} \PY{n}{x}\PY{p}{:} \PY{n+nb}{abs}\PY{p}{(}\PY{n}{x} \PY{o}{\PYZhy{}} \PY{n}{lambda\PYZus{}ev}\PY{p}{)}\PY{p}{)}
        \PY{n}{frac\PYZus{}lambda\PYZus{}ev} \PY{o}{=} \PY{n}{fractional\PYZus{}eigenvalue\PYZus{}convert}\PY{p}{(}\PY{n}{lambda\PYZus{}eigenvalues}\PY{p}{[}\PY{n}{lambda\PYZus{}ev}\PY{p}{]}\PY{p}{)}
        \PY{k}{if} \PY{n}{eigenvalue\PYZus{}oracle}\PY{p}{[}\PY{n}{closest\PYZus{}exp\PYZus{}ev}\PY{p}{]} \PY{o}{\PYZlt{}} \PY{l+m+mf}{0.0}\PY{p}{:} \PY{c+c1}{\PYZsh{} if the eigenvalue is negative we must invert it}
            \PY{n}{frac\PYZus{}lambda\PYZus{}ev} \PY{o}{=} \PY{o}{\PYZhy{}}\PY{l+m+mi}{1} \PY{o}{*} \PY{p}{(}\PY{l+m+mi}{1} \PY{o}{\PYZhy{}} \PY{n}{frac\PYZus{}lambda\PYZus{}ev}\PY{p}{)}
        
        \PY{n}{lambda\PYZus{}j} \PY{o}{=} \PY{n}{frac\PYZus{}lambda\PYZus{}ev} \PY{o}{*} \PY{p}{(}\PY{l+m+mi}{2}\PY{o}{*}\PY{n}{pi}\PY{o}{/}\PY{n}{t}\PY{p}{)}
        \PY{n}{C} \PY{o}{=} \PY{n+nb}{min}\PY{p}{(}\PY{n}{C} \PY{k}{if} \PY{n}{C} \PY{o+ow}{is} \PY{o+ow}{not} \PY{k+kc}{None} \PY{k}{else} \PY{n+nb}{abs}\PY{p}{(}\PY{n}{lambda\PYZus{}j}\PY{p}{)}\PY{p}{,} \PY{n+nb}{abs}\PY{p}{(}\PY{n}{lambda\PYZus{}j}\PY{p}{)}\PY{o}{\PYZhy{}}\PY{l+m+mf}{0.0001}\PY{p}{)}
        \PY{n}{lambda\PYZus{}js}\PY{o}{.}\PY{n}{append}\PY{p}{(}\PY{n}{lambda\PYZus{}j}\PY{p}{)}
    
    \PY{n}{thetas} \PY{o}{=} \PY{p}{[}\PY{p}{]}
    
    \PY{k}{if} \PY{n}{normal}\PY{p}{:}
        \PY{k}{for} \PY{n}{lambda\PYZus{}j} \PY{o+ow}{in} \PY{n}{lambda\PYZus{}js}\PY{p}{:}
            \PY{n}{thetas}\PY{o}{.}\PY{n}{append}\PY{p}{(}\PY{l+m+mi}{2}\PY{o}{*}\PY{n}{np}\PY{o}{.}\PY{n}{arcsin}\PY{p}{(}\PY{n}{C}\PY{o}{/}\PY{n}{lambda\PYZus{}j}\PY{p}{)}\PY{p}{)}
    \PY{k}{else}\PY{p}{:}
        \PY{k}{for} \PY{n}{lambda\PYZus{}j} \PY{o+ow}{in} \PY{n}{lambda\PYZus{}js}\PY{p}{:}
            \PY{n}{thetas}\PY{o}{.}\PY{n}{append}\PY{p}{(}\PY{l+m+mi}{2}\PY{o}{*}\PY{n}{np}\PY{o}{.}\PY{n}{arcsin}\PY{p}{(}\PY{n}{C}\PY{o}{*}\PY{n}{lambda\PYZus{}j}\PY{p}{)}\PY{p}{)}
    \PY{n+nb}{print}\PY{p}{(}\PY{l+s+s1}{\PYZsq{}}\PY{l+s+s1}{Rotation Angles:}\PY{l+s+s1}{\PYZsq{}}\PY{p}{,} \PY{n}{np}\PY{o}{.}\PY{n}{array2string}\PY{p}{(}\PY{n}{np}\PY{o}{.}\PY{n}{array}\PY{p}{(}\PY{n}{thetas}\PY{p}{)}\PY{p}{,} \PY{n}{formatter}\PY{o}{=}\PY{p}{\PYZob{}}\PY{l+s+s1}{\PYZsq{}}\PY{l+s+s1}{float\PYZus{}kind}\PY{l+s+s1}{\PYZsq{}}\PY{p}{:} \PY{k}{lambda} \PY{n}{x}\PY{p}{:} \PY{l+s+sa}{f}\PY{l+s+s2}{\PYZdq{}}\PY{l+s+si}{\PYZob{}}\PY{n}{x}\PY{l+s+si}{:}\PY{l+s+s2}{.2f}\PY{l+s+si}{\PYZcb{}}\PY{l+s+s2}{\PYZdq{}}\PY{p}{\PYZcb{}}\PY{p}{)}\PY{p}{)}
    \PY{k}{return} \PY{n}{np}\PY{o}{.}\PY{n}{array}\PY{p}{(}\PY{n}{thetas}\PY{p}{)}

\PY{k}{def} \PY{n+nf}{fractional\PYZus{}eigenvalue\PYZus{}convert}\PY{p}{(}
        \PY{n}{bitstring}\PY{p}{:} \PY{n}{np}\PY{o}{.}\PY{n}{ndarray}\PY{p}{,}
\PY{p}{)}\PY{p}{:}
    \PY{n}{frac} \PY{o}{=} \PY{l+m+mi}{0}
    \PY{k}{for} \PY{n}{ind}\PY{p}{,} \PY{n}{i} \PY{o+ow}{in} \PY{n+nb}{enumerate}\PY{p}{(}\PY{n}{bitstring}\PY{p}{[}\PY{p}{:}\PY{p}{:}\PY{l+m+mi}{1}\PY{p}{]}\PY{p}{)}\PY{p}{:}
        \PY{n}{frac} \PY{o}{+}\PY{o}{=} \PY{n+nb}{int}\PY{p}{(}\PY{n}{i}\PY{p}{)} \PY{o}{/} \PY{l+m+mi}{2}\PY{o}{*}\PY{o}{*}\PY{p}{(}\PY{n}{ind}\PY{o}{+}\PY{l+m+mi}{1}\PY{p}{)}
    \PY{k}{return} \PY{n}{frac}
\end{Verbatim}
\end{tcolorbox}

    The mother-function that creates the HHL circuit given the problem
statement and certain hyperparameters.

    \begin{tcolorbox}[breakable, size=fbox, boxrule=1pt, pad at break*=1mm,colback=cellbackground, colframe=cellborder]
\prompt{In}{incolor}{7}{\boxspacing}
\begin{Verbatim}[commandchars=\\\{\}]
\PY{c+c1}{\PYZsh{}\PYZsh{}\PYZsh{}\PYZsh{}\PYZsh{}\PYZsh{}\PYZsh{}\PYZsh{}\PYZsh{}\PYZsh{}\PYZsh{}\PYZsh{}\PYZsh{}\PYZsh{}\PYZsh{}\PYZsh{}\PYZsh{}\PYZsh{}\PYZsh{}\PYZsh{}\PYZsh{}\PYZsh{}\PYZsh{}\PYZsh{}\PYZsh{}\PYZsh{}\PYZsh{}\PYZsh{}\PYZsh{}\PYZsh{}\PYZsh{}\PYZsh{}\PYZsh{}\PYZsh{}\PYZsh{}\PYZsh{}\PYZsh{}\PYZsh{}\PYZsh{}\PYZsh{}\PYZsh{}\PYZsh{}\PYZsh{}\PYZsh{}\PYZsh{}\PYZsh{}\PYZsh{}\PYZsh{}\PYZsh{}\PYZsh{}\PYZsh{}\PYZsh{}\PYZsh{}\PYZsh{}}
\PY{c+c1}{\PYZsh{}\PYZsh{}\PYZsh{}\PYZsh{}\PYZsh{}\PYZsh{}\PYZsh{}\PYZsh{}\PYZsh{}\PYZsh{}\PYZsh{}\PYZsh{}\PYZsh{}\PYZsh{}\PYZsh{}\PYZsh{}\PYZsh{}\PYZsh{}\PYZsh{} HHL Circuit \PYZsh{}\PYZsh{}\PYZsh{}\PYZsh{}\PYZsh{}\PYZsh{}\PYZsh{}\PYZsh{}\PYZsh{}\PYZsh{}\PYZsh{}\PYZsh{}\PYZsh{}\PYZsh{}\PYZsh{}\PYZsh{}\PYZsh{}\PYZsh{}\PYZsh{}\PYZsh{}\PYZsh{}\PYZsh{}}
\PY{c+c1}{\PYZsh{}\PYZsh{}\PYZsh{}\PYZsh{}\PYZsh{}\PYZsh{}\PYZsh{}\PYZsh{}\PYZsh{}\PYZsh{}\PYZsh{}\PYZsh{}\PYZsh{}\PYZsh{}\PYZsh{}\PYZsh{}\PYZsh{}\PYZsh{}\PYZsh{}\PYZsh{}\PYZsh{}\PYZsh{}\PYZsh{}\PYZsh{}\PYZsh{}\PYZsh{}\PYZsh{}\PYZsh{}\PYZsh{}\PYZsh{}\PYZsh{}\PYZsh{}\PYZsh{}\PYZsh{}\PYZsh{}\PYZsh{}\PYZsh{}\PYZsh{}\PYZsh{}\PYZsh{}\PYZsh{}\PYZsh{}\PYZsh{}\PYZsh{}\PYZsh{}\PYZsh{}\PYZsh{}\PYZsh{}\PYZsh{}\PYZsh{}\PYZsh{}\PYZsh{}\PYZsh{}\PYZsh{}}
\PY{k}{def} \PY{n+nf}{hhl\PYZus{}circuit}\PY{p}{(}
        \PY{n}{A}\PY{p}{:} \PY{n}{np}\PY{o}{.}\PY{n}{ndarray}\PY{p}{,}
        \PY{n}{b}\PY{p}{:} \PY{n}{np}\PY{o}{.}\PY{n}{ndarray}\PY{p}{,}
        \PY{n}{tol}\PY{p}{:} \PY{n+nb}{float} \PY{o}{=} \PY{l+m+mf}{1e\PYZhy{}1}\PY{p}{,}
        \PY{n}{t}\PY{p}{:} \PY{n+nb}{float} \PY{o}{=} \PY{n}{pi}\PY{p}{,}
        \PY{n}{measure}\PY{p}{:} \PY{n+nb}{bool} \PY{o}{=} \PY{k+kc}{False}\PY{p}{,}
\PY{p}{)} \PY{o}{\PYZhy{}}\PY{o}{\PYZgt{}} \PY{n}{QuantumCircuit}\PY{p}{:}
\PY{+w}{    }\PY{l+s+sd}{\PYZdq{}\PYZdq{}\PYZdq{}    }
\PY{l+s+sd}{    :param A: A numpy matrix }
\PY{l+s+sd}{    :param b: A numpy vector}
\PY{l+s+sd}{    :return: A quantum circuit for the hhl algorithm}
\PY{l+s+sd}{    \PYZdq{}\PYZdq{}\PYZdq{}}
    
    \PY{c+c1}{\PYZsh{}\PYZsh{}\PYZsh{}\PYZsh{} PREPARE REGISTERS + PARAMS \PYZsh{}\PYZsh{}\PYZsh{}\PYZsh{}}
    \PY{n}{n\PYZus{}og} \PY{o}{=} \PY{n}{b}\PY{o}{.}\PY{n}{shape}\PY{p}{[}\PY{l+m+mi}{0}\PY{p}{]}
    \PY{n}{A}\PY{p}{,} \PY{n}{b}\PY{p}{,} \PY{n}{U}\PY{p}{,} \PY{n}{nb}\PY{p}{,} \PY{n}{nq} \PY{o}{=} \PY{n}{prepare\PYZus{}hhlparams}\PY{p}{(}\PY{n}{A}\PY{p}{,} \PY{n}{b}\PY{p}{,} \PY{n}{tol}\PY{p}{,} \PY{n}{ev\PYZus{}time}\PY{p}{)}
    \PY{n}{n\PYZus{}b} \PY{o}{=} \PY{n}{QuantumRegister}\PY{p}{(}\PY{n}{nb}\PY{p}{,} \PY{l+s+s1}{\PYZsq{}}\PY{l+s+s1}{b}\PY{l+s+s1}{\PYZsq{}}\PY{p}{)}
    \PY{n}{n\PYZus{}q} \PY{o}{=} \PY{n}{QuantumRegister}\PY{p}{(}\PY{n}{nq}\PY{p}{,} \PY{l+s+s1}{\PYZsq{}}\PY{l+s+s1}{QPE}\PY{l+s+s1}{\PYZsq{}}\PY{p}{)}
    \PY{n}{n\PYZus{}a} \PY{o}{=} \PY{n}{QuantumRegister}\PY{p}{(}\PY{l+m+mi}{1}\PY{p}{,} \PY{l+s+s1}{\PYZsq{}}\PY{l+s+s1}{aux}\PY{l+s+s1}{\PYZsq{}}\PY{p}{)}
    \PY{n}{cb} \PY{o}{=} \PY{n}{ClassicalRegister}\PY{p}{(}\PY{p}{(}\PY{l+m+mi}{1}\PY{o}{+}\PY{n}{nq}\PY{o}{+}\PY{n}{nb}\PY{p}{)}\PY{p}{,}\PY{l+s+s1}{\PYZsq{}}\PY{l+s+s1}{cb}\PY{l+s+s1}{\PYZsq{}}\PY{p}{)}
  
    \PY{c+c1}{\PYZsh{}\PYZsh{}\PYZsh{}\PYZsh{} INITIALIZE B \PYZsh{}\PYZsh{}\PYZsh{}\PYZsh{}}
    \PY{n}{init\PYZus{}circ} \PY{o}{=} \PY{n}{initb\PYZus{}construct\PYZus{}circuit}\PY{p}{(}\PY{n}{n\PYZus{}b}\PY{p}{,} \PY{n}{n\PYZus{}q}\PY{p}{,} \PY{n}{n\PYZus{}a}\PY{p}{,} \PY{n}{cb}\PY{p}{,} \PY{n}{b}\PY{p}{)}
    \PY{n}{init\PYZus{}circ}\PY{o}{.}\PY{n}{barrier}\PY{p}{(}\PY{p}{)}
    
    \PY{c+c1}{\PYZsh{}\PYZsh{}\PYZsh{}\PYZsh{} QPE CIRCUIT \PYZsh{}\PYZsh{}\PYZsh{}\PYZsh{}}
    \PY{n}{qpe\PYZus{}circ} \PY{o}{=} \PY{n}{QPE\PYZus{}construct\PYZus{}circuit}\PY{p}{(}\PY{n}{n\PYZus{}b}\PY{p}{,} \PY{n}{n\PYZus{}q}\PY{p}{,} \PY{n}{n\PYZus{}a}\PY{p}{,} \PY{n}{cb}\PY{p}{,} \PY{n}{U}\PY{p}{)}
    \PY{n}{qpe\PYZus{}circ}\PY{o}{.}\PY{n}{barrier}\PY{p}{(}\PY{p}{)}
    
    \PY{c+c1}{\PYZsh{}\PYZsh{}\PYZsh{}\PYZsh{} EIGENVALUE ROTATION CIRCUIT \PYZsh{}\PYZsh{}\PYZsh{}\PYZsh{}}
    \PY{n}{eigrot\PYZus{}circ} \PY{o}{=} \PY{n}{eigrot\PYZus{}construct\PYZus{}circuit}\PY{p}{(}\PY{n}{b}\PY{p}{,} \PY{n}{A}\PY{p}{,} \PY{n}{U}\PY{p}{,} \PY{n}{n\PYZus{}b}\PY{p}{,} \PY{n}{n\PYZus{}q}\PY{p}{,} \PY{n}{n\PYZus{}a}\PY{p}{,} \PY{n}{cb}\PY{p}{,} \PY{n}{t}\PY{p}{,} \PY{k+kc}{True}\PY{p}{)}
    \PY{n}{eigrot\PYZus{}circ}\PY{o}{.}\PY{n}{barrier}\PY{p}{(}\PY{p}{)}
    
    \PY{c+c1}{\PYZsh{}\PYZsh{}\PYZsh{}\PYZsh{} MEASURING ANCILLA \PYZsh{}\PYZsh{}\PYZsh{}\PYZsh{}}
    \PY{n}{measure\PYZus{}anc\PYZus{}circ} \PY{o}{=} \PY{n}{QuantumCircuit}\PY{p}{(}\PY{n}{n\PYZus{}b}\PY{p}{,} \PY{n}{n\PYZus{}q}\PY{p}{,} \PY{n}{n\PYZus{}a}\PY{p}{,} \PY{n}{cb}\PY{p}{)}
    \PY{n}{measure\PYZus{}anc\PYZus{}circ}\PY{o}{.}\PY{n}{measure}\PY{p}{(}\PY{n}{n\PYZus{}a}\PY{p}{,} \PY{n}{cb}\PY{p}{[}\PY{l+m+mi}{0}\PY{p}{]}\PY{p}{)}
    \PY{n}{measure\PYZus{}anc\PYZus{}circ}\PY{o}{.}\PY{n}{barrier}\PY{p}{(}\PY{p}{)}
    
    \PY{c+c1}{\PYZsh{}\PYZsh{}\PYZsh{}\PYZsh{} UN\PYZhy{}COMPUTING QPE CIRCUIT \PYZsh{}\PYZsh{}\PYZsh{}\PYZsh{}}
    \PY{n}{qpe\PYZus{}dag\PYZus{}circ} \PY{o}{=} \PY{n}{qpe\PYZus{}circ}\PY{o}{.}\PY{n}{inverse}\PY{p}{(}\PY{p}{)}
    \PY{n}{qpe\PYZus{}dag\PYZus{}circ}\PY{o}{.}\PY{n}{barrier}\PY{p}{(}\PY{p}{)}
    
    \PY{c+c1}{\PYZsh{}\PYZsh{}\PYZsh{}\PYZsh{} MEASURING QPE Registers \PYZsh{}\PYZsh{}\PYZsh{}\PYZsh{}}
    \PY{n}{measureQPE\PYZus{}circ} \PY{o}{=} \PY{n}{QuantumCircuit}\PY{p}{(}\PY{n}{n\PYZus{}b}\PY{p}{,} \PY{n}{n\PYZus{}q}\PY{p}{,} \PY{n}{n\PYZus{}a}\PY{p}{,} \PY{n}{cb}\PY{p}{)}
    \PY{k}{if} \PY{n}{measure}\PY{p}{:}
        \PY{n}{measureQPE\PYZus{}circ}\PY{o}{.}\PY{n}{measure}\PY{p}{(}\PY{n}{n\PYZus{}q}\PY{p}{,} \PY{n}{cb}\PY{p}{[}\PY{l+m+mi}{1}\PY{p}{:}\PY{n}{nq}\PY{o}{+}\PY{l+m+mi}{1}\PY{p}{]}\PY{p}{)}
    
    \PY{c+c1}{\PYZsh{}\PYZsh{}\PYZsh{}\PYZsh{} MEASURING B \PYZsh{}\PYZsh{}\PYZsh{}\PYZsh{}}
    \PY{n}{measureb\PYZus{}circ} \PY{o}{=} \PY{n}{QuantumCircuit}\PY{p}{(}\PY{n}{n\PYZus{}b}\PY{p}{,} \PY{n}{n\PYZus{}q}\PY{p}{,} \PY{n}{n\PYZus{}a}\PY{p}{,} \PY{n}{cb}\PY{p}{)}
    \PY{k}{if} \PY{n}{measure}\PY{p}{:}
        \PY{n}{measureb\PYZus{}circ}\PY{o}{.}\PY{n}{measure}\PY{p}{(}\PY{n}{n\PYZus{}b}\PY{p}{,} \PY{n}{cb}\PY{p}{[}\PY{n}{nq}\PY{o}{+}\PY{l+m+mi}{1}\PY{p}{:}\PY{p}{]}\PY{p}{)}
    
    \PY{c+c1}{\PYZsh{}\PYZsh{}\PYZsh{}\PYZsh{} COMPOSING THE CIRCUIT \PYZsh{}\PYZsh{}\PYZsh{}\PYZsh{}}
    \PY{n}{circuit} \PY{o}{=} \PY{n}{init\PYZus{}circ}\PY{o}{.}\PY{n}{compose}\PY{p}{(}\PY{n}{qpe\PYZus{}circ}\PY{p}{)}
    \PY{n}{circuit} \PY{o}{=} \PY{n}{circuit}\PY{o}{.}\PY{n}{compose}\PY{p}{(}\PY{n}{eigrot\PYZus{}circ}\PY{p}{)}
    \PY{n}{circuit} \PY{o}{=} \PY{n}{circuit}\PY{o}{.}\PY{n}{compose}\PY{p}{(}\PY{n}{measure\PYZus{}anc\PYZus{}circ}\PY{p}{)}
    \PY{n}{circuit} \PY{o}{=} \PY{n}{circuit}\PY{o}{.}\PY{n}{compose}\PY{p}{(}\PY{n}{qpe\PYZus{}dag\PYZus{}circ}\PY{p}{)}
    \PY{n}{circuit} \PY{o}{=} \PY{n}{circuit}\PY{o}{.}\PY{n}{compose}\PY{p}{(}\PY{n}{measureQPE\PYZus{}circ}\PY{p}{)}
    \PY{n}{circuit} \PY{o}{=} \PY{n}{circuit}\PY{o}{.}\PY{n}{compose}\PY{p}{(}\PY{n}{measureb\PYZus{}circ}\PY{p}{)}

    \PY{k}{return} \PY{n}{A}\PY{p}{,} \PY{n}{b}\PY{p}{,} \PY{n}{nq}\PY{p}{,} \PY{n}{circuit}
\end{Verbatim}
\end{tcolorbox}

    \subsubsection{Section 4: Results}\label{section-4-results}

Finally, we can test our circuit given a problem statement (\(A\),
\(b\)) that fulfills the requirements listed in Section 1. In addition,
we give the function a tolerance (tol: for QPE's approximation
accuracy), evolution time (t), and the number of execution shots.

You can see below the specific 3x3 invertible non hermitian matrix we
have chosen to use.

    \begin{tcolorbox}[breakable, size=fbox, boxrule=1pt, pad at break*=1mm,colback=cellbackground, colframe=cellborder]
\prompt{In}{incolor}{8}{\boxspacing}
\begin{Verbatim}[commandchars=\\\{\}]
\PY{c+c1}{\PYZsh{}\PYZsh{}\PYZsh{}\PYZsh{}\PYZsh{}\PYZsh{}\PYZsh{}\PYZsh{}\PYZsh{}\PYZsh{}\PYZsh{}\PYZsh{}\PYZsh{}\PYZsh{}\PYZsh{}\PYZsh{}\PYZsh{}\PYZsh{}\PYZsh{}\PYZsh{}\PYZsh{}\PYZsh{}\PYZsh{}\PYZsh{}\PYZsh{}\PYZsh{}\PYZsh{}\PYZsh{}\PYZsh{}\PYZsh{}\PYZsh{}\PYZsh{}\PYZsh{}\PYZsh{}\PYZsh{}\PYZsh{}\PYZsh{}\PYZsh{}\PYZsh{}\PYZsh{}\PYZsh{}\PYZsh{}\PYZsh{}\PYZsh{}\PYZsh{}\PYZsh{}\PYZsh{}\PYZsh{}\PYZsh{}\PYZsh{}\PYZsh{}\PYZsh{}\PYZsh{}\PYZsh{}}
\PY{c+c1}{\PYZsh{}\PYZsh{}\PYZsh{}\PYZsh{}\PYZsh{}\PYZsh{}\PYZsh{}\PYZsh{}\PYZsh{}\PYZsh{}\PYZsh{}\PYZsh{}\PYZsh{}\PYZsh{} Hyperparams + Problem \PYZsh{}\PYZsh{}\PYZsh{}\PYZsh{}\PYZsh{}\PYZsh{}\PYZsh{}\PYZsh{}\PYZsh{}\PYZsh{}\PYZsh{}\PYZsh{}\PYZsh{}\PYZsh{}\PYZsh{}\PYZsh{}\PYZsh{}}
\PY{c+c1}{\PYZsh{}\PYZsh{}\PYZsh{}\PYZsh{}\PYZsh{}\PYZsh{}\PYZsh{}\PYZsh{}\PYZsh{}\PYZsh{}\PYZsh{}\PYZsh{}\PYZsh{}\PYZsh{}\PYZsh{}\PYZsh{}\PYZsh{}\PYZsh{}\PYZsh{}\PYZsh{}\PYZsh{}\PYZsh{}\PYZsh{}\PYZsh{}\PYZsh{}\PYZsh{}\PYZsh{}\PYZsh{}\PYZsh{}\PYZsh{}\PYZsh{}\PYZsh{}\PYZsh{}\PYZsh{}\PYZsh{}\PYZsh{}\PYZsh{}\PYZsh{}\PYZsh{}\PYZsh{}\PYZsh{}\PYZsh{}\PYZsh{}\PYZsh{}\PYZsh{}\PYZsh{}\PYZsh{}\PYZsh{}\PYZsh{}\PYZsh{}\PYZsh{}\PYZsh{}\PYZsh{}\PYZsh{}}
\PY{n}{shots} \PY{o}{=} \PY{l+m+mi}{2}\PY{o}{*}\PY{o}{*}\PY{l+m+mi}{16} \PY{c+c1}{\PYZsh{} execution shots}

\PY{c+c1}{\PYZsh{} Try these problem statements out, each with carefully chosen hyperparameters}

\PY{c+c1}{\PYZsh{} A is a 3x3 invertible non hermitian sparse matrix that is well scaled and stable \PYZhy{} GOOD EX}
\PY{c+c1}{\PYZsh{} tol = 2e\PYZhy{}6}
\PY{c+c1}{\PYZsh{} ev\PYZus{}time = 150}
\PY{c+c1}{\PYZsh{} A = np.matrix([[1, 1, 0],[\PYZhy{}1, 1, \PYZhy{}1],[0, 0.12, 1.2]])}
\PY{c+c1}{\PYZsh{} b = np.array([1, 2, 3])}

\PY{c+c1}{\PYZsh{} A is a 3x3 invertible non hermitian sparse matrix that is well scaled and stable \PYZhy{} OKAY EX}
\PY{c+c1}{\PYZsh{} tol = 2e\PYZhy{}4}
\PY{c+c1}{\PYZsh{} ev\PYZus{}time = 2*pi}
\PY{c+c1}{\PYZsh{} A = np.matrix([[4/5, 1/5, 0],[0, 3/5, 1/5],[0, 0, 2.9/5]])}
\PY{c+c1}{\PYZsh{} b = np.array([1, 2, 3])}

\PY{c+c1}{\PYZsh{} A is a 2x2 hermitian matrix}
\PY{c+c1}{\PYZsh{} tol = 1e\PYZhy{}1}
\PY{c+c1}{\PYZsh{} ev\PYZus{}time = pi}
\PY{c+c1}{\PYZsh{} A = np.matrix([[3/4, 1/4],[1/4, 3/4]])}
\PY{c+c1}{\PYZsh{} b = np.array([0, 1])}

\PY{c+c1}{\PYZsh{} A is a complex 4x4 hermitian matrix}
\PY{n}{tol} \PY{o}{=} \PY{l+m+mf}{1e\PYZhy{}4}
\PY{n}{ev\PYZus{}time} \PY{o}{=} \PY{n}{pi}
\PY{n}{A} \PY{o}{=} \PY{n}{np}\PY{o}{.}\PY{n}{matrix}\PY{p}{(}\PY{p}{[}\PY{p}{[}\PY{l+m+mi}{4}\PY{p}{,} \PY{l+m+mi}{1} \PY{o}{+} \PY{l+m+mi}{1}\PY{n}{j}\PY{p}{,} \PY{l+m+mi}{0}\PY{p}{,} \PY{l+m+mi}{0}\PY{p}{]}\PY{p}{,}\PY{p}{[}\PY{l+m+mi}{1} \PY{o}{\PYZhy{}} \PY{l+m+mi}{1}\PY{n}{j}\PY{p}{,} \PY{l+m+mi}{4}\PY{p}{,} \PY{l+m+mf}{0.5}\PY{p}{,} \PY{l+m+mi}{0}\PY{p}{]}\PY{p}{,}\PY{p}{[}\PY{l+m+mi}{0}\PY{p}{,} \PY{l+m+mf}{0.5}\PY{p}{,} \PY{l+m+mi}{4}\PY{p}{,} \PY{l+m+mf}{0.5}\PY{p}{]}\PY{p}{,}\PY{p}{[}\PY{l+m+mi}{0}\PY{p}{,} \PY{l+m+mi}{0}\PY{p}{,} \PY{l+m+mf}{0.5}\PY{p}{,} \PY{l+m+mi}{4}\PY{p}{]}\PY{p}{]}\PY{p}{)}
\PY{n}{b} \PY{o}{=} \PY{n}{np}\PY{o}{.}\PY{n}{array}\PY{p}{(}\PY{p}{[}\PY{l+m+mi}{0}\PY{p}{,} \PY{l+m+mi}{1}\PY{p}{,} \PY{l+m+mi}{0}\PY{p}{,} \PY{l+m+mi}{1}\PY{p}{]}\PY{p}{)}

\PY{c+c1}{\PYZsh{}\PYZsh{}\PYZsh{}\PYZsh{}\PYZsh{}\PYZsh{}\PYZsh{}\PYZsh{}\PYZsh{}\PYZsh{}\PYZsh{}\PYZsh{}\PYZsh{}\PYZsh{}\PYZsh{}\PYZsh{}\PYZsh{}\PYZsh{}\PYZsh{}\PYZsh{}\PYZsh{}\PYZsh{}\PYZsh{}\PYZsh{}\PYZsh{}\PYZsh{}\PYZsh{}\PYZsh{}\PYZsh{}\PYZsh{}\PYZsh{}\PYZsh{}\PYZsh{}\PYZsh{}\PYZsh{}\PYZsh{}\PYZsh{}\PYZsh{}\PYZsh{}\PYZsh{}\PYZsh{}\PYZsh{}\PYZsh{}\PYZsh{}\PYZsh{}\PYZsh{}\PYZsh{}\PYZsh{}\PYZsh{}\PYZsh{}\PYZsh{}\PYZsh{}\PYZsh{}\PYZsh{}}
\PY{c+c1}{\PYZsh{}\PYZsh{}\PYZsh{}\PYZsh{}\PYZsh{}\PYZsh{}\PYZsh{}\PYZsh{}\PYZsh{}\PYZsh{}\PYZsh{}\PYZsh{}\PYZsh{}\PYZsh{}\PYZsh{}\PYZsh{}\PYZsh{}\PYZsh{}\PYZsh{} HHL Creation \PYZsh{}\PYZsh{}\PYZsh{}\PYZsh{}\PYZsh{}\PYZsh{}\PYZsh{}\PYZsh{}\PYZsh{}\PYZsh{}\PYZsh{}\PYZsh{}\PYZsh{}\PYZsh{}\PYZsh{}\PYZsh{}\PYZsh{}\PYZsh{}\PYZsh{}\PYZsh{}\PYZsh{}}
\PY{c+c1}{\PYZsh{}\PYZsh{}\PYZsh{}\PYZsh{}\PYZsh{}\PYZsh{}\PYZsh{}\PYZsh{}\PYZsh{}\PYZsh{}\PYZsh{}\PYZsh{}\PYZsh{}\PYZsh{}\PYZsh{}\PYZsh{}\PYZsh{}\PYZsh{}\PYZsh{}\PYZsh{}\PYZsh{}\PYZsh{}\PYZsh{}\PYZsh{}\PYZsh{}\PYZsh{}\PYZsh{}\PYZsh{}\PYZsh{}\PYZsh{}\PYZsh{}\PYZsh{}\PYZsh{}\PYZsh{}\PYZsh{}\PYZsh{}\PYZsh{}\PYZsh{}\PYZsh{}\PYZsh{}\PYZsh{}\PYZsh{}\PYZsh{}\PYZsh{}\PYZsh{}\PYZsh{}\PYZsh{}\PYZsh{}\PYZsh{}\PYZsh{}\PYZsh{}\PYZsh{}\PYZsh{}\PYZsh{}}

\PY{n}{A\PYZus{}p}\PY{p}{,} \PY{n}{b\PYZus{}p}\PY{p}{,} \PY{n}{nq}\PY{p}{,} \PY{n}{hhl} \PY{o}{=} \PY{n}{hhl\PYZus{}circuit}\PY{p}{(}\PY{n}{A}\PY{p}{,} \PY{n}{b}\PY{p}{,} \PY{n}{tol}\PY{p}{,} \PY{n}{ev\PYZus{}time}\PY{p}{,} \PY{k+kc}{True}\PY{p}{)}
\PY{n}{hhl}\PY{o}{.}\PY{n}{draw}\PY{p}{(}\PY{l+s+s1}{\PYZsq{}}\PY{l+s+s1}{mpl}\PY{l+s+s1}{\PYZsq{}}\PY{p}{,} \PY{n}{reverse\PYZus{}bits}\PY{o}{=}\PY{k+kc}{True}\PY{p}{,} \PY{n}{scale}\PY{o}{=}\PY{l+m+mf}{0.2}\PY{p}{,} \PY{n}{fold}\PY{o}{=}\PY{l+m+mi}{100}\PY{p}{)}
\end{Verbatim}
\end{tcolorbox}

    \begin{Verbatim}[commandchars=\\\{\}]
Eigenvalues of A': [0.452 0.641 0.811 1.000]
Condition number of A': 2.213140406338677
    \end{Verbatim}

    \begin{Verbatim}[commandchars=\\\{\}]
/opt/anaconda3/envs/ece396\_2024/lib/python3.11/site-
packages/numpy/linalg/linalg.py:2180: RuntimeWarning: divide by zero encountered
in det
  r = \_umath\_linalg.det(a, signature=signature)
/opt/anaconda3/envs/ece396\_2024/lib/python3.11/site-
packages/numpy/linalg/linalg.py:2180: RuntimeWarning: invalid value encountered
in det
  r = \_umath\_linalg.det(a, signature=signature)
    \end{Verbatim}

    \begin{Verbatim}[commandchars=\\\{\}]
Rotation Angles: [0.94 3.10 1.57 1.18]
    \end{Verbatim}
 
            
\prompt{Out}{outcolor}{8}{}
    
    \begin{center}
    \adjustimage{max size={0.9\linewidth}{0.9\paperheight}}{QuantumLinearAlgebra_HHL_Final_files/QuantumLinearAlgebra_HHL_Final_19_3.png}
    \end{center}
    { \hspace*{\fill} \\}
    

    \begin{tcolorbox}[breakable, size=fbox, boxrule=1pt, pad at break*=1mm,colback=cellbackground, colframe=cellborder]
\prompt{In}{incolor}{9}{\boxspacing}
\begin{Verbatim}[commandchars=\\\{\}]
\PY{c+c1}{\PYZsh{}\PYZsh{}\PYZsh{}\PYZsh{}\PYZsh{}\PYZsh{}\PYZsh{}\PYZsh{}\PYZsh{}\PYZsh{}\PYZsh{}\PYZsh{}\PYZsh{}\PYZsh{}\PYZsh{}\PYZsh{}\PYZsh{}\PYZsh{}\PYZsh{}\PYZsh{}\PYZsh{}\PYZsh{}\PYZsh{}\PYZsh{}\PYZsh{}\PYZsh{}\PYZsh{}\PYZsh{}\PYZsh{}\PYZsh{}\PYZsh{}\PYZsh{}\PYZsh{}\PYZsh{}\PYZsh{}\PYZsh{}\PYZsh{}\PYZsh{}\PYZsh{}\PYZsh{}\PYZsh{}\PYZsh{}\PYZsh{}\PYZsh{}\PYZsh{}\PYZsh{}\PYZsh{}\PYZsh{}\PYZsh{}\PYZsh{}\PYZsh{}\PYZsh{}\PYZsh{}\PYZsh{}}
\PY{c+c1}{\PYZsh{}\PYZsh{}\PYZsh{}\PYZsh{}\PYZsh{}\PYZsh{}\PYZsh{}\PYZsh{}\PYZsh{}\PYZsh{}\PYZsh{}\PYZsh{}\PYZsh{}\PYZsh{}\PYZsh{}\PYZsh{}\PYZsh{}\PYZsh{}\PYZsh{}\PYZsh{}\PYZsh{} Results \PYZsh{}\PYZsh{}\PYZsh{}\PYZsh{}\PYZsh{}\PYZsh{}\PYZsh{}\PYZsh{}\PYZsh{}\PYZsh{}\PYZsh{}\PYZsh{}\PYZsh{}\PYZsh{}\PYZsh{}\PYZsh{}\PYZsh{}\PYZsh{}\PYZsh{}\PYZsh{}\PYZsh{}\PYZsh{}\PYZsh{}\PYZsh{}}
\PY{c+c1}{\PYZsh{}\PYZsh{}\PYZsh{}\PYZsh{}\PYZsh{}\PYZsh{}\PYZsh{}\PYZsh{}\PYZsh{}\PYZsh{}\PYZsh{}\PYZsh{}\PYZsh{}\PYZsh{}\PYZsh{}\PYZsh{}\PYZsh{}\PYZsh{}\PYZsh{}\PYZsh{}\PYZsh{}\PYZsh{}\PYZsh{}\PYZsh{}\PYZsh{}\PYZsh{}\PYZsh{}\PYZsh{}\PYZsh{}\PYZsh{}\PYZsh{}\PYZsh{}\PYZsh{}\PYZsh{}\PYZsh{}\PYZsh{}\PYZsh{}\PYZsh{}\PYZsh{}\PYZsh{}\PYZsh{}\PYZsh{}\PYZsh{}\PYZsh{}\PYZsh{}\PYZsh{}\PYZsh{}\PYZsh{}\PYZsh{}\PYZsh{}\PYZsh{}\PYZsh{}\PYZsh{}\PYZsh{}}

\PY{n}{job} \PY{o}{=} \PY{n}{execute}\PY{p}{(}\PY{n}{hhl}\PY{p}{,} \PY{n}{backend}\PY{o}{=}\PY{n}{backend\PYZus{}qasm}\PY{p}{,} \PY{n}{shots}\PY{o}{=}\PY{n}{shots}\PY{p}{)}
\PY{n}{counts} \PY{o}{=} \PY{n}{job}\PY{o}{.}\PY{n}{result}\PY{p}{(}\PY{p}{)}\PY{o}{.}\PY{n}{get\PYZus{}counts}\PY{p}{(}\PY{p}{)}
\PY{n}{failed\PYZus{}count}\PY{p}{,} \PY{n}{trimmed\PYZus{}counts} \PY{o}{=} \PY{n}{remove\PYZus{}fails}\PY{p}{(}\PY{n}{counts}\PY{o}{.}\PY{n}{copy}\PY{p}{(}\PY{p}{)}\PY{p}{)}

\PY{n}{x\PYZus{}HHL} \PY{o}{=} \PY{n}{get\PYZus{}x\PYZus{}distribution\PYZus{}hhl}\PY{p}{(}\PY{n}{shots}\PY{p}{,} \PY{n}{failed\PYZus{}count}\PY{p}{,} \PY{n}{trimmed\PYZus{}counts}\PY{p}{,} \PY{n}{b}\PY{p}{)}
\PY{n}{x} \PY{o}{=} \PY{n}{get\PYZus{}x\PYZus{}distribution\PYZus{}actual}\PY{p}{(}\PY{n}{A}\PY{p}{,} \PY{n}{b}\PY{p}{)}
\PY{n+nb}{print}\PY{p}{(}\PY{l+s+sa}{f}\PY{l+s+s2}{\PYZdq{}}\PY{l+s+s2}{The HHL circuit failed }\PY{l+s+si}{\PYZob{}}\PY{n}{failed\PYZus{}count}\PY{o}{/}\PY{n}{shots}\PY{o}{*}\PY{l+m+mi}{100}\PY{l+s+si}{\PYZcb{}}\PY{l+s+s2}{\PYZpc{} of the time.}\PY{l+s+s2}{\PYZdq{}}\PY{p}{)}
\PY{n+nb}{print}\PY{p}{(}\PY{l+s+s1}{\PYZsq{}}\PY{l+s+s1}{|x\PYZgt{} prob. HHL   :}\PY{l+s+s1}{\PYZsq{}}\PY{p}{,} \PY{n}{np}\PY{o}{.}\PY{n}{array2string}\PY{p}{(}\PY{n}{x\PYZus{}HHL}\PY{p}{,} \PY{n}{formatter}\PY{o}{=}\PY{p}{\PYZob{}}\PY{l+s+s1}{\PYZsq{}}\PY{l+s+s1}{float\PYZus{}kind}\PY{l+s+s1}{\PYZsq{}}\PY{p}{:} \PY{k}{lambda} \PY{n}{x}\PY{p}{:} \PY{l+s+sa}{f}\PY{l+s+s2}{\PYZdq{}}\PY{l+s+si}{\PYZob{}}\PY{n}{x}\PY{l+s+si}{:}\PY{l+s+s2}{.3f}\PY{l+s+si}{\PYZcb{}}\PY{l+s+s2}{\PYZdq{}}\PY{p}{\PYZcb{}}\PY{p}{)}\PY{p}{)}
\PY{n+nb}{print}\PY{p}{(}\PY{l+s+s1}{\PYZsq{}}\PY{l+s+s1}{|x\PYZgt{} prob. actual:}\PY{l+s+s1}{\PYZsq{}}\PY{p}{,} \PY{n}{np}\PY{o}{.}\PY{n}{array2string}\PY{p}{(}\PY{n}{np}\PY{o}{.}\PY{n}{real}\PY{p}{(}\PY{n}{x}\PY{p}{)}\PY{p}{,} \PY{n}{formatter}\PY{o}{=}\PY{p}{\PYZob{}}\PY{l+s+s1}{\PYZsq{}}\PY{l+s+s1}{float\PYZus{}kind}\PY{l+s+s1}{\PYZsq{}}\PY{p}{:} \PY{k}{lambda} \PY{n}{x}\PY{p}{:} \PY{l+s+sa}{f}\PY{l+s+s2}{\PYZdq{}}\PY{l+s+si}{\PYZob{}}\PY{n}{x}\PY{l+s+si}{:}\PY{l+s+s2}{.3f}\PY{l+s+si}{\PYZcb{}}\PY{l+s+s2}{\PYZdq{}}\PY{p}{\PYZcb{}}\PY{p}{)}\PY{p}{)}
\PY{n+nb}{print}\PY{p}{(}\PY{l+s+s1}{\PYZsq{}}\PY{l+s+s1}{Normalized Error:}\PY{l+s+s1}{\PYZsq{}}\PY{p}{,} \PY{n}{np}\PY{o}{.}\PY{n}{linalg}\PY{o}{.}\PY{n}{norm}\PY{p}{(}\PY{n}{x\PYZus{}HHL} \PY{o}{\PYZhy{}} \PY{n}{x}\PY{p}{)}\PY{p}{)}

\PY{c+c1}{\PYZsh{} plot\PYZus{}histogram(counts) \PYZsh{} qiskit\PYZsq{}s visualizer}

\PY{n}{fig}\PY{p}{,} \PY{n}{ax} \PY{o}{=} \PY{n}{plt}\PY{o}{.}\PY{n}{subplots}\PY{p}{(}\PY{n}{figsize}\PY{o}{=}\PY{p}{(}\PY{l+m+mi}{10}\PY{p}{,} \PY{l+m+mi}{6}\PY{p}{)}\PY{p}{)} \PY{c+c1}{\PYZsh{} matplotlib}
\PY{n}{ax}\PY{o}{.}\PY{n}{set\PYZus{}facecolor}\PY{p}{(}\PY{l+s+s1}{\PYZsq{}}\PY{l+s+s1}{\PYZsh{}f2f2f2}\PY{l+s+s1}{\PYZsq{}}\PY{p}{)}
\PY{n}{keys} \PY{o}{=} \PY{n+nb}{list}\PY{p}{(}\PY{n}{counts}\PY{o}{.}\PY{n}{keys}\PY{p}{(}\PY{p}{)}\PY{p}{)}
\PY{n}{values} \PY{o}{=} \PY{n+nb}{list}\PY{p}{(}\PY{n}{counts}\PY{o}{.}\PY{n}{values}\PY{p}{(}\PY{p}{)}\PY{p}{)}
\PY{n}{plt}\PY{o}{.}\PY{n}{rcParams}\PY{o}{.}\PY{n}{update}\PY{p}{(}\PY{p}{\PYZob{}}\PY{l+s+s1}{\PYZsq{}}\PY{l+s+s1}{font.family}\PY{l+s+s1}{\PYZsq{}}\PY{p}{:} \PY{l+s+s1}{\PYZsq{}}\PY{l+s+s1}{Times New Roman}\PY{l+s+s1}{\PYZsq{}}\PY{p}{\PYZcb{}}\PY{p}{)}
\PY{n}{ax}\PY{o}{.}\PY{n}{bar}\PY{p}{(}\PY{n}{keys}\PY{p}{,} \PY{n}{values}\PY{p}{)}
\PY{n}{ax}\PY{o}{.}\PY{n}{tick\PYZus{}params}\PY{p}{(}\PY{n}{axis}\PY{o}{=}\PY{l+s+s1}{\PYZsq{}}\PY{l+s+s1}{x}\PY{l+s+s1}{\PYZsq{}}\PY{p}{,} \PY{n}{labelsize}\PY{o}{=}\PY{l+m+mi}{8}\PY{p}{,} \PY{n}{rotation}\PY{o}{=}\PY{l+m+mi}{0}\PY{p}{)}
\PY{n}{ax}\PY{o}{.}\PY{n}{set\PYZus{}xticks}\PY{p}{(}\PY{n+nb}{range}\PY{p}{(}\PY{n+nb}{len}\PY{p}{(}\PY{n}{keys}\PY{p}{)}\PY{p}{)}\PY{p}{)}
\PY{n}{keys\PYZus{}formatted} \PY{o}{=} \PY{p}{[}\PY{l+s+s1}{\PYZsq{}}\PY{l+s+se}{\PYZbs{}n}\PY{l+s+s1}{\PYZsq{}}\PY{o}{.}\PY{n}{join}\PY{p}{(}\PY{n+nb}{list}\PY{p}{(}\PY{n}{key}\PY{p}{)}\PY{p}{)} \PY{k}{for} \PY{n}{key} \PY{o+ow}{in} \PY{n}{keys}\PY{p}{]}
\PY{n}{ax}\PY{o}{.}\PY{n}{set\PYZus{}xticklabels}\PY{p}{(}\PY{n}{keys\PYZus{}formatted}\PY{p}{)}
\PY{n}{ax}\PY{o}{.}\PY{n}{set\PYZus{}xlabel}\PY{p}{(}\PY{l+s+s1}{\PYZsq{}}\PY{l+s+s1}{Final States}\PY{l+s+s1}{\PYZsq{}}\PY{p}{)}
\PY{n}{ax}\PY{o}{.}\PY{n}{set\PYZus{}ylabel}\PY{p}{(}\PY{l+s+s1}{\PYZsq{}}\PY{l+s+s1}{Counts}\PY{l+s+s1}{\PYZsq{}}\PY{p}{)}
\PY{n}{ax}\PY{o}{.}\PY{n}{set\PYZus{}title}\PY{p}{(}\PY{l+s+s1}{\PYZsq{}}\PY{l+s+s1}{HHL Circuit Results}\PY{l+s+s1}{\PYZsq{}}\PY{p}{)}

\PY{n}{plt}\PY{o}{.}\PY{n}{tight\PYZus{}layout}\PY{p}{(}\PY{p}{)}
\PY{n}{plt}\PY{o}{.}\PY{n}{show}\PY{p}{(}\PY{p}{)}
\end{Verbatim}
\end{tcolorbox}

    \begin{Verbatim}[commandchars=\\\{\}]
The HHL circuit failed 67.80853271484375\% of the time.
|x> prob. HHL   : [0.006 0.527 0.063 0.404]
|x> prob. actual: [0.000 0.514 0.028 0.394]
Normalized Error: 0.07513430378501756
    \end{Verbatim}

    \begin{center}
    \adjustimage{max size={0.9\linewidth}{0.9\paperheight}}{QuantumLinearAlgebra_HHL_Final_files/QuantumLinearAlgebra_HHL_Final_20_1.png}
    \end{center}
    { \hspace*{\fill} \\}
    
    From our results, you can see the final states retrieved after executing
the circuit for the specified number of shots. It is important to note
which shots correspond to success and which correspond to failure
(specified in Section 1). The results are displayed with the least
significant bit (ancilla bit) displayed at the bottom, followed by the
QPE register bits stacked on top, with the last \(\log_2(n)\) bits on
top.

In addition, you can see the difference between the probability
distributions of our \(\ket{x}\) from the HHL algorithm, and a generic
classical numpy algorithm.

It is important to note that as HHL is quite sensitive to the matrix
\(A\) itself, the success of the results also depends on the matrix. A
good rule of thumb is that high failure rates often indicate inaccurate
results. Also, changing the hyperparameters such as evolution time also
depend on each matrix. If the two dimensions of \(\ket{x}\) mismatch,
the circuit failed, run again.

    \subsubsection{Section 5: Subroutines}\label{section-5-subroutines}

The following section will now perform a data fitting algorithm outlined
by N. Wiebe, D. Braun, and S. Lloyd
\href{https://journals.aps.org/prl/abstract/10.1103/PhysRevLett.109.050505\#fulltext}{{[}DOI{]}}.
HHL is used as a subroutine within this larger data fitting problem,
which is broken down into three subroutines: (1) Fitting Psuedoinverse
(2) Estimating Fit Quality (3) Learning \(\boldsymbol{\lambda}\)

Our problem is defined as minimizing
\[E = |\mathbf{F}\boldsymbol{\lambda} - \mathbf{y}|^2, \quad\text{ such that } \boldsymbol{\lambda} \in \mathbb{C}^N \text{ and } \mathbf{F} \in \mathbb{C}^{N \times M} \text{ and } \mathbf{y} \in \mathbb{C}^{M}\]
by choosing the minimizing fit parameters \(\boldsymbol{\lambda}\)
(Least squares fitting problem). We also will assume the conditions that
\(\frac{1}{\kappa^2} \leq \| F^{\dagger} F \| \leq 1\) and
\(\frac{1}{\kappa^2} \leq \| F F^{\dagger}\| \leq 1\) where \(\kappa\)
is the condition number of the invertible matrix \(\mathbf{F}\) and
\(\|\cdot\|\) denotes the spectral norm.

Input: A Quantum State: \(\ket{y}\), the condition number of
\(\mathbf{F}\): \(\kappa\), the sparseness of \(\mathbf{F}\): s, and an
error tolerance value: \(\epsilon\).

For our purposes, we will solve a data fitting problem where
\(\mathbf{F} \in \mathbb{R}^{4\times3}\) with rank 3, meaning
\(M=4, N=3\). In addition, \(y \in \mathbb{R}^4\) that is not a linear
combination of the columns of F. In addition, note that the
Moore-Penrose pseudoinverse of \(\mathbf{F}\) is denoted
\(\mathbf{F}^+\) to \(y\) is
\[\boldsymbol{\lambda} = \mathbf{F}^+\mathbf{y} = (\mathbf{F}^\dagger \mathbf{F})^{-1} \mathbf{F}^\dagger \mathbf{y}\]
and will be used to solve the fitting problem.

    \begin{tcolorbox}[breakable, size=fbox, boxrule=1pt, pad at break*=1mm,colback=cellbackground, colframe=cellborder]
\prompt{In}{incolor}{105}{\boxspacing}
\begin{Verbatim}[commandchars=\\\{\}]
\PY{n}{tol} \PY{o}{=} \PY{l+m+mf}{1e\PYZhy{}4}
\PY{n}{ev\PYZus{}time} \PY{o}{=} \PY{l+m+mi}{150}
\PY{n}{shots} \PY{o}{=} \PY{l+m+mi}{1024}
\PY{n}{F} \PY{o}{=} \PY{n}{np}\PY{o}{.}\PY{n}{matrix}\PY{p}{(}\PY{p}{[}\PY{p}{[}\PY{l+m+mi}{1}\PY{p}{,} \PY{l+m+mi}{1}\PY{p}{,} \PY{l+m+mi}{1}\PY{p}{]}\PY{p}{,} \PY{p}{[}\PY{l+m+mi}{0}\PY{p}{,} \PY{l+m+mi}{0}\PY{p}{,} \PY{l+m+mi}{1}\PY{p}{]}\PY{p}{,} \PY{p}{[}\PY{l+m+mi}{1}\PY{p}{,} \PY{l+m+mi}{0}\PY{p}{,} \PY{l+m+mi}{0}\PY{p}{]}\PY{p}{,} \PY{p}{[}\PY{l+m+mf}{0.5}\PY{p}{,} \PY{l+m+mi}{1}\PY{p}{,} \PY{l+m+mi}{0}\PY{p}{]}\PY{p}{]}\PY{p}{)}
\PY{n}{y} \PY{o}{=} \PY{n}{np}\PY{o}{.}\PY{n}{array}\PY{p}{(}\PY{p}{[}\PY{l+m+mi}{0}\PY{p}{,} \PY{l+m+mi}{0}\PY{p}{,} \PY{l+m+mi}{1}\PY{p}{,} \PY{l+m+mi}{0}\PY{p}{]}\PY{p}{)}
\end{Verbatim}
\end{tcolorbox}

    \paragraph{\texorpdfstring{\textbf{1. The
Pseudoinverse}}{1. The Pseudoinverse}}\label{the-pseudoinverse}

Given the general Moore-Penrose pseudoinverse
\(\lambda = F^+y = (F^\textdagger F)^{-1} F^\textdagger y\), we can
reformulate the components to be written as
\[(F^\textdagger F) \lambda =  F^\textdagger y.\] This has the same
structure as the linear system \(Ax = b\), where
\(A = (F^\textdagger F)\) and \(b = F^\textdagger y\). Therefore, we can
apply the HHL algorithm to estimate \(\lambda\) after some
pre-processing of our matrix, vector inputs. As implemented above, HHL
can handle non-hermitian matrices by forming the specfied block matrix
in section 2.

As stated in the article, this first sub-algorithm will take in the
hyperparameters, then using the trick defined above, will perform a call
to HHL in order to retrieve the estimated
\(\ket{\boldsymbol{\lambda}}\). To see this first subroutine in action
with statistical analysis, skip to the final subroutine that performs
the psuedoinverse and analysis. The function below is to be used in the
swap test (subroutine 2).

    \begin{tcolorbox}[breakable, size=fbox, boxrule=1pt, pad at break*=1mm,colback=cellbackground, colframe=cellborder]
\prompt{In}{incolor}{98}{\boxspacing}
\begin{Verbatim}[commandchars=\\\{\}]
\PY{k}{def} \PY{n+nf}{psuedoinverse\PYZus{}hhl}\PY{p}{(}
    \PY{n}{F}\PY{p}{:} \PY{n}{np}\PY{o}{.}\PY{n}{ndarray}\PY{p}{,}
    \PY{n}{y}\PY{p}{:} \PY{n}{np}\PY{o}{.}\PY{n}{ndarray}\PY{p}{,}
    \PY{n}{tol}\PY{p}{:} \PY{n+nb}{float}\PY{p}{,}
    \PY{n}{ev\PYZus{}time}\PY{p}{:} \PY{n+nb}{float}\PY{p}{,}
\PY{p}{)} \PY{o}{\PYZhy{}}\PY{o}{\PYZgt{}} \PY{p}{(}\PY{n}{np}\PY{o}{.}\PY{n}{ndarray}\PY{p}{,} \PY{n}{np}\PY{o}{.}\PY{n}{ndarray}\PY{p}{)}\PY{p}{:}
    \PY{n}{F\PYZus{}dag} \PY{o}{=} \PY{n}{F}\PY{o}{.}\PY{n}{getH}\PY{p}{(}\PY{p}{)}
    \PY{n}{A} \PY{o}{=} \PY{n}{F\PYZus{}dag} \PY{o}{@} \PY{n}{F}
    \PY{n}{b} \PY{o}{=} \PY{n}{np}\PY{o}{.}\PY{n}{dot}\PY{p}{(}\PY{n}{F\PYZus{}dag}\PY{p}{,} \PY{n}{y}\PY{o}{.}\PY{n}{T}\PY{p}{)}
    \PY{n}{b} \PY{o}{=} \PY{n}{np}\PY{o}{.}\PY{n}{asarray}\PY{p}{(}\PY{n}{b}\PY{o}{.}\PY{n}{flatten}\PY{p}{(}\PY{p}{)}\PY{p}{)}\PY{p}{[}\PY{l+m+mi}{0}\PY{p}{]}
    
    \PY{n}{A\PYZus{}p}\PY{p}{,} \PY{n}{b\PYZus{}p}\PY{p}{,} \PY{n}{nq}\PY{p}{,} \PY{n}{hhl} \PY{o}{=} \PY{n}{hhl\PYZus{}circuit}\PY{p}{(}\PY{n}{A}\PY{p}{,} \PY{n}{b}\PY{p}{,} \PY{n}{tol}\PY{p}{,} \PY{n}{ev\PYZus{}time}\PY{p}{,} \PY{k+kc}{False}\PY{p}{)}
    \PY{n}{nb} \PY{o}{=} \PY{n+nb}{int}\PY{p}{(}\PY{n}{np}\PY{o}{.}\PY{n}{ceil}\PY{p}{(}\PY{n}{np}\PY{o}{.}\PY{n}{log2}\PY{p}{(}\PY{n+nb}{len}\PY{p}{(}\PY{n}{b\PYZus{}p}\PY{p}{)}\PY{p}{)}\PY{p}{)}\PY{p}{)}

    \PY{c+c1}{\PYZsh{} returns the new linear system A, b}
    \PY{c+c1}{\PYZsh{} number of qubits for the solution register}
    \PY{c+c1}{\PYZsh{} and a circuit where the last nb registers correspond to the encoded |\PYZbs{}lambda\PYZgt{}}
    \PY{k}{return} \PY{n}{A}\PY{p}{,} \PY{n}{b}\PY{p}{,} \PY{n}{nb}\PY{p}{,} \PY{n}{hhl}
    
\end{Verbatim}
\end{tcolorbox}

    \paragraph{\texorpdfstring{\textbf{2. Estimating Fit
Quality}}{2. Estimating Fit Quality}}\label{estimating-fit-quality}

Ultimately, to estimate fit quality, we can then perform the Swap Test
to distinguish \(\ket{y}\) and \(I(F)\ket{\lambda}\) controlled by a
qubit in the \(\ket{+}\) state. If the ancilla is measured to be in 1,
we can conclude that the states are different, otherwise, they are the
same. Since from our HHL, we have a state in \(\ket{\lambda}\), we must
now apply a variant of the HHL algorithm to get \(I(F)\ket{\lambda}\).
The variant occurs in the eigenvalue rotation step, where instead of
rotation angles that are \(\theta_j = 2\arcsin(C/\tilde{\lambda_j})\),
are \(\theta_j = 2\arcsin(C\tilde{\lambda_j})\) where \(C\) is not the
maximum eigenvalue because we want to multiply by the eigenvalues of
\(I(F)\). This derivation is outlined in the article, but the general
intuition is that instead of applying the inverse of a given matrix to
the register as in HHL, we are applying the matrix itself.

Note: \(I(\cdot)\) corresponds to making the isometry (same as the block
matrix operation in Section 2).

\textbf{For our purposes}, because this algorithm is quite involved and
as the article so aptly puts it, ``is not straightforward,'' an
analogous path we will take is to utilize our hhl circuit to retrieve
\(\ket{\lambda}_{hhl}\), then we can classically compute our expected
\(\ket{\lambda}_{exp}\), and perform the swap test on the two states.

    \begin{tcolorbox}[breakable, size=fbox, boxrule=1pt, pad at break*=1mm,colback=cellbackground, colframe=cellborder]
\prompt{In}{incolor}{99}{\boxspacing}
\begin{Verbatim}[commandchars=\\\{\}]
\PY{k}{def} \PY{n+nf}{isometry\PYZus{}application\PYZus{}circuit}\PY{p}{(}
    \PY{n}{F}\PY{p}{:} \PY{n}{np}\PY{o}{.}\PY{n}{ndarray}\PY{p}{,}
    \PY{n}{y}\PY{p}{:} \PY{n}{np}\PY{o}{.}\PY{n}{ndarray}\PY{p}{,}
    \PY{n}{tol}\PY{p}{:} \PY{n+nb}{float}\PY{p}{,}
    \PY{n}{t}\PY{p}{:} \PY{n+nb}{float}\PY{p}{,}
\PY{p}{)}\PY{p}{:}
    \PY{n}{nq} \PY{o}{=} \PY{n+nb}{int}\PY{p}{(}\PY{n}{np}\PY{o}{.}\PY{n}{log}\PY{p}{(}\PY{l+m+mi}{1}\PY{o}{/}\PY{n}{tol}\PY{p}{)}\PY{p}{)} 
    \PY{n}{A}\PY{p}{,} \PY{n}{b}\PY{p}{,} \PY{n}{nb}\PY{p}{,} \PY{n}{hhl} \PY{o}{=} \PY{n}{psuedoinverse\PYZus{}hhl}\PY{p}{(}\PY{n}{F}\PY{p}{,} \PY{n}{y}\PY{p}{,} \PY{n}{tol}\PY{p}{,} \PY{n}{t}\PY{p}{)}
    
    \PY{c+c1}{\PYZsh{}\PYZsh{}\PYZsh{}\PYZsh{} CALCULATING CLASSICALLY \PYZsh{}\PYZsh{}\PYZsh{}\PYZsh{}}
    \PY{n}{N} \PY{o}{=} \PY{l+m+mi}{2}\PY{o}{*}\PY{o}{*}\PY{n}{nb}
    \PY{n}{exp\PYZus{}lambda} \PY{o}{=} \PY{n}{np}\PY{o}{.}\PY{n}{linalg}\PY{o}{.}\PY{n}{solve}\PY{p}{(}\PY{n}{A}\PY{p}{,} \PY{n}{b}\PY{p}{)}
    \PY{n}{exp\PYZus{}lambda\PYZus{}pad} \PY{o}{=} \PY{n}{np}\PY{o}{.}\PY{n}{pad}\PY{p}{(}\PY{n}{exp\PYZus{}lambda}\PY{p}{,} \PY{p}{(}\PY{l+m+mi}{0}\PY{p}{,} \PY{n}{N} \PY{o}{\PYZhy{}} \PY{n+nb}{len}\PY{p}{(}\PY{n}{exp\PYZus{}lambda}\PY{p}{)}\PY{p}{)}\PY{p}{)}
    
    \PY{n}{n\PYZus{}a\PYZus{}p} \PY{o}{=} \PY{n}{QuantumRegister}\PY{p}{(}\PY{l+m+mi}{1}\PY{p}{,} \PY{l+s+s1}{\PYZsq{}}\PY{l+s+s1}{swap\PYZus{}aux}\PY{l+s+s1}{\PYZsq{}}\PY{p}{)}
    \PY{n}{n\PYZus{}b\PYZus{}p} \PY{o}{=} \PY{n}{QuantumRegister}\PY{p}{(}\PY{n}{nb}\PY{p}{,} \PY{l+s+s1}{\PYZsq{}}\PY{l+s+s1}{b\PYZus{}p}\PY{l+s+s1}{\PYZsq{}}\PY{p}{)}
    \PY{n}{hhl}\PY{o}{.}\PY{n}{add\PYZus{}register}\PY{p}{(}\PY{n}{n\PYZus{}a\PYZus{}p}\PY{p}{)}
    \PY{n}{hhl}\PY{o}{.}\PY{n}{add\PYZus{}register}\PY{p}{(}\PY{n}{n\PYZus{}b\PYZus{}p}\PY{p}{)}
    
    \PY{c+c1}{\PYZsh{}\PYZsh{}\PYZsh{}\PYZsh{} INITIALIZE LAMBDA\PYZus{}PRIME \PYZsh{}\PYZsh{}\PYZsh{}\PYZsh{}}
    \PY{n}{hhl}\PY{o}{.}\PY{n}{initialize}\PY{p}{(}\PY{n}{exp\PYZus{}lambda\PYZus{}pad}\PY{o}{/}\PY{n}{np}\PY{o}{.}\PY{n}{linalg}\PY{o}{.}\PY{n}{norm}\PY{p}{(}\PY{n}{exp\PYZus{}lambda\PYZus{}pad}\PY{p}{)}\PY{p}{,} \PY{n}{n\PYZus{}b\PYZus{}p}\PY{p}{)}
    
    \PY{c+c1}{\PYZsh{}\PYZsh{}\PYZsh{}\PYZsh{} PERFORM SWAP TEST \PYZsh{}\PYZsh{}\PYZsh{}\PYZsh{}}
    \PY{n}{hhl}\PY{o}{.}\PY{n}{h}\PY{p}{(}\PY{n}{n\PYZus{}a\PYZus{}p}\PY{p}{)}
    \PY{k}{for} \PY{n}{i} \PY{o+ow}{in} \PY{n+nb}{range}\PY{p}{(}\PY{n}{nb}\PY{p}{)}\PY{p}{:}
        \PY{n}{hhl}\PY{o}{.}\PY{n}{cswap}\PY{p}{(}\PY{n}{n\PYZus{}a\PYZus{}p}\PY{p}{,} \PY{n}{i}\PY{p}{,} \PY{n}{n\PYZus{}b\PYZus{}p}\PY{p}{[}\PY{n}{i}\PY{p}{]}\PY{p}{)}
    \PY{n}{hhl}\PY{o}{.}\PY{n}{h}\PY{p}{(}\PY{n}{n\PYZus{}a\PYZus{}p}\PY{p}{)}
    \PY{n}{hhl}\PY{o}{.}\PY{n}{measure}\PY{p}{(}\PY{n}{n\PYZus{}a\PYZus{}p}\PY{p}{,} \PY{l+m+mi}{1}\PY{p}{)}
    \PY{n}{hhl}\PY{o}{.}\PY{n}{barrier}\PY{p}{(}\PY{p}{)}
    
    \PY{c+c1}{\PYZsh{}\PYZsh{}\PYZsh{} MEASURE QPE REGISTERS FOR SUCCESS \PYZsh{}\PYZsh{}\PYZsh{}}
    \PY{n}{hhl}\PY{o}{.}\PY{n}{measure}\PY{p}{(}\PY{n+nb}{list}\PY{p}{(}\PY{n+nb}{range}\PY{p}{(}\PY{n}{nb}\PY{p}{,}\PY{n}{nb}\PY{o}{+}\PY{n}{nq}\PY{p}{)}\PY{p}{)}\PY{p}{,} \PY{n+nb}{list}\PY{p}{(}\PY{n+nb}{range}\PY{p}{(}\PY{l+m+mi}{2}\PY{p}{,}\PY{n}{nq}\PY{o}{+}\PY{l+m+mi}{2}\PY{p}{)}\PY{p}{)}\PY{p}{)}
    
    \PY{k}{return} \PY{n}{hhl}
    
\end{Verbatim}
\end{tcolorbox}

    \begin{tcolorbox}[breakable, size=fbox, boxrule=1pt, pad at break*=1mm,colback=cellbackground, colframe=cellborder]
\prompt{In}{incolor}{100}{\boxspacing}
\begin{Verbatim}[commandchars=\\\{\}]
\PY{n}{est\PYZus{}fitqual\PYZus{}circ} \PY{o}{=} \PY{n}{isometry\PYZus{}application\PYZus{}circuit}\PY{p}{(}\PY{n}{F}\PY{p}{,} \PY{n}{y}\PY{p}{,} \PY{n}{tol}\PY{p}{,} \PY{n}{ev\PYZus{}time}\PY{p}{)}
\PY{n}{est\PYZus{}fitqual\PYZus{}circ}\PY{o}{.}\PY{n}{draw}\PY{p}{(}\PY{l+s+s1}{\PYZsq{}}\PY{l+s+s1}{mpl}\PY{l+s+s1}{\PYZsq{}}\PY{p}{,} \PY{n}{reverse\PYZus{}bits}\PY{o}{=}\PY{k+kc}{True}\PY{p}{,} \PY{n}{scale}\PY{o}{=}\PY{l+m+mf}{0.2}\PY{p}{,} \PY{n}{fold}\PY{o}{=}\PY{l+m+mi}{100}\PY{p}{)}
\end{Verbatim}
\end{tcolorbox}

    \begin{Verbatim}[commandchars=\\\{\}]
Eigenvalues of A': [0.138 0.225 0.267 1.000]
Condition number of A': 7.2299263194302705
Rotation Angles: [0.27 3.07 2.66 1.08]
    \end{Verbatim}
 
            
\prompt{Out}{outcolor}{100}{}
    
    \begin{center}
    \adjustimage{max size={0.9\linewidth}{0.9\paperheight}}{QuantumLinearAlgebra_HHL_Final_files/QuantumLinearAlgebra_HHL_Final_28_1.png}
    \end{center}
    { \hspace*{\fill} \\}
    

    Now, out of the successful HHL runs (meaning the first ancilla is 1 and
the QPEs are all 0), we will count the frequency of our swap acilla
being 1. Given how we set up the measurements, we are looking for when
the measurement has the state \(11 + 0*nq\) divided by all successful
runs.

    \begin{tcolorbox}[breakable, size=fbox, boxrule=1pt, pad at break*=1mm,colback=cellbackground, colframe=cellborder]
\prompt{In}{incolor}{101}{\boxspacing}
\begin{Verbatim}[commandchars=\\\{\}]
\PY{n}{res} \PY{o}{=} \PY{n}{execute}\PY{p}{(}\PY{n}{est\PYZus{}fitqual\PYZus{}circ}\PY{p}{,} \PY{n}{backend\PYZus{}qasm}\PY{p}{,} \PY{n}{shots}\PY{o}{=}\PY{n}{shots}\PY{p}{)}\PY{o}{.}\PY{n}{result}\PY{p}{(}\PY{p}{)}
\PY{n}{counts} \PY{o}{=} \PY{n}{res}\PY{o}{.}\PY{n}{get\PYZus{}counts}\PY{p}{(}\PY{p}{)}
\end{Verbatim}
\end{tcolorbox}

    \begin{tcolorbox}[breakable, size=fbox, boxrule=1pt, pad at break*=1mm,colback=cellbackground, colframe=cellborder]
\prompt{In}{incolor}{103}{\boxspacing}
\begin{Verbatim}[commandchars=\\\{\}]
\PY{c+c1}{\PYZsh{} remove all failed runs: either ancilla is |0\PYZgt{} or our QPE register is not all |0\PYZgt{}}
\PY{n}{test} \PY{o}{=} \PY{n+nb}{list}\PY{p}{(}\PY{n}{counts}\PY{o}{.}\PY{n}{keys}\PY{p}{(}\PY{p}{)}\PY{p}{)}\PY{p}{[}\PY{l+m+mi}{0}\PY{p}{]}
\PY{n}{nq} \PY{o}{=} \PY{n+nb}{int}\PY{p}{(}\PY{n}{np}\PY{o}{.}\PY{n}{log}\PY{p}{(}\PY{l+m+mi}{1}\PY{o}{/}\PY{n}{tol}\PY{p}{)}\PY{p}{)}
\PY{n}{failed\PYZus{}count} \PY{o}{=} \PY{n+nb}{sum}\PY{p}{(}\PY{n}{value} \PY{k}{for} \PY{n}{key}\PY{p}{,} \PY{n}{value} \PY{o+ow}{in} \PY{n}{counts}\PY{o}{.}\PY{n}{items}\PY{p}{(}\PY{p}{)} \PY{k}{if} \PY{p}{(}\PY{n}{key}\PY{o}{.}\PY{n}{endswith}\PY{p}{(}\PY{l+s+s2}{\PYZdq{}}\PY{l+s+s2}{0}\PY{l+s+s2}{\PYZdq{}}\PY{p}{)} \PY{p}{)}\PY{p}{)}
\PY{n}{successful\PYZus{}count} \PY{o}{=} \PY{n}{shots} \PY{o}{\PYZhy{}} \PY{n}{failed\PYZus{}count}

\PY{n}{fails} \PY{o}{=} \PY{p}{[}\PY{p}{]}
\PY{n}{only\PYZus{}successes} \PY{o}{=} \PY{n}{counts}\PY{o}{.}\PY{n}{copy}\PY{p}{(}\PY{p}{)}
\PY{k}{for} \PY{n}{key}\PY{p}{,} \PY{n}{val} \PY{o+ow}{in} \PY{n}{counts}\PY{o}{.}\PY{n}{items}\PY{p}{(}\PY{p}{)}\PY{p}{:}
    \PY{k}{if} \PY{n}{key}\PY{o}{.}\PY{n}{endswith}\PY{p}{(}\PY{l+s+s2}{\PYZdq{}}\PY{l+s+s2}{0}\PY{l+s+s2}{\PYZdq{}}\PY{p}{)} \PY{o+ow}{or} \PY{n}{key}\PY{p}{[}\PY{o}{\PYZhy{}}\PY{l+m+mi}{2}\PY{p}{]} \PY{o}{!=} \PY{l+s+s2}{\PYZdq{}}\PY{l+s+s2}{1}\PY{l+s+s2}{\PYZdq{}}\PY{p}{:}
        \PY{n}{fails}\PY{o}{.}\PY{n}{append}\PY{p}{(}\PY{n}{key}\PY{p}{)}
\PY{k}{for} \PY{n}{item} \PY{o+ow}{in} \PY{n}{fails}\PY{p}{:}
    \PY{n}{only\PYZus{}successes}\PY{o}{.}\PY{n}{pop}\PY{p}{(}\PY{n}{item}\PY{p}{)}

\PY{n}{matching} \PY{o}{=} \PY{n+nb}{sum}\PY{p}{(}\PY{n}{value} \PY{k}{for} \PY{n}{key}\PY{p}{,} \PY{n}{value} \PY{o+ow}{in} \PY{n}{only\PYZus{}successes}\PY{o}{.}\PY{n}{items}\PY{p}{(}\PY{p}{)}\PY{p}{)}
\PY{n+nb}{print}\PY{p}{(}\PY{l+s+s1}{\PYZsq{}}\PY{l+s+s1}{Percentage of actual matchings:}\PY{l+s+s1}{\PYZsq{}}\PY{p}{,} \PY{n}{matching}\PY{o}{/}\PY{n}{successful\PYZus{}count} \PY{o}{*} \PY{l+m+mi}{100}\PY{p}{)}
\PY{n+nb}{print}\PY{p}{(}\PY{n}{only\PYZus{}successes}\PY{p}{)}
\end{Verbatim}
\end{tcolorbox}

    \begin{Verbatim}[commandchars=\\\{\}]
Percentage of actual matchings: 1.6085790884718498
\{'001100000011': 1, '000000000011': 3, '010000000011': 2\}
    \end{Verbatim}

    Thus, we have indicated that our circuit runs successfully to achieve a
solution of close enough fidelity to our classically computed answer
1.60\% of the successful instances (using the swap test). This is quite
low, but is not completely unexpected as a complete match is extremely
hard with the inherent numerical errors present in an HHL algorithm that
is query taxing. Keep in mind, this value is determinant on the
tolerance we set, as well as hyperparameter tuning the evolution time.
Finally, we do expect a lower accuracy as the condition number of the
matrix being used is quite large for the HHL algorithm.

Nevertheless, performing HHL can still provide valuable insight onto the
fitting's footprint, as demonstrated in the final subroutine.

    \paragraph{\texorpdfstring{\textbf{3. Learning
Lambda}}{3. Learning Lambda}}\label{learning-lambda}

The Final subroutine is learning lambda through statistical sampling,
similar to what was achieved when analyzing the results of our HHL
implementation.

    \begin{tcolorbox}[breakable, size=fbox, boxrule=1pt, pad at break*=1mm,colback=cellbackground, colframe=cellborder]
\prompt{In}{incolor}{106}{\boxspacing}
\begin{Verbatim}[commandchars=\\\{\}]
\PY{c+c1}{\PYZsh{} Performing Pseudoinverse Step}
\PY{n}{F\PYZus{}dag} \PY{o}{=} \PY{n}{F}\PY{o}{.}\PY{n}{getH}\PY{p}{(}\PY{p}{)}
\PY{n}{A} \PY{o}{=} \PY{n}{F\PYZus{}dag} \PY{o}{@} \PY{n}{F}
\PY{n}{b} \PY{o}{=} \PY{n}{np}\PY{o}{.}\PY{n}{dot}\PY{p}{(}\PY{n}{F\PYZus{}dag}\PY{p}{,} \PY{n}{y}\PY{o}{.}\PY{n}{T}\PY{p}{)}
\PY{n}{b} \PY{o}{=} \PY{n}{np}\PY{o}{.}\PY{n}{asarray}\PY{p}{(}\PY{n}{b}\PY{o}{.}\PY{n}{flatten}\PY{p}{(}\PY{p}{)}\PY{p}{)}\PY{p}{[}\PY{l+m+mi}{0}\PY{p}{]}

\PY{c+c1}{\PYZsh{} Constructing hhl circuit with measurement}
\PY{n}{A\PYZus{}p}\PY{p}{,} \PY{n}{b\PYZus{}p}\PY{p}{,} \PY{n}{nq}\PY{p}{,} \PY{n}{hhl} \PY{o}{=} \PY{n}{hhl\PYZus{}circuit}\PY{p}{(}\PY{n}{A}\PY{p}{,} \PY{n}{b}\PY{p}{,} \PY{n}{tol}\PY{p}{,} \PY{n}{ev\PYZus{}time}\PY{p}{,} \PY{k+kc}{True}\PY{p}{)}

\PY{c+c1}{\PYZsh{} Executing and Analyzing}
\PY{n}{job} \PY{o}{=} \PY{n}{execute}\PY{p}{(}\PY{n}{hhl}\PY{p}{,} \PY{n}{backend}\PY{o}{=}\PY{n}{backend\PYZus{}qasm}\PY{p}{,} \PY{n}{shots}\PY{o}{=}\PY{n}{shots}\PY{p}{)}
\PY{n}{counts} \PY{o}{=} \PY{n}{job}\PY{o}{.}\PY{n}{result}\PY{p}{(}\PY{p}{)}\PY{o}{.}\PY{n}{get\PYZus{}counts}\PY{p}{(}\PY{p}{)}
\PY{n}{failed\PYZus{}count}\PY{p}{,} \PY{n}{trimmed\PYZus{}counts} \PY{o}{=} \PY{n}{remove\PYZus{}fails}\PY{p}{(}\PY{n}{counts}\PY{o}{.}\PY{n}{copy}\PY{p}{(}\PY{p}{)}\PY{p}{)}

\PY{n}{x\PYZus{}HHL} \PY{o}{=} \PY{n}{get\PYZus{}x\PYZus{}distribution\PYZus{}hhl}\PY{p}{(}\PY{n}{shots}\PY{p}{,} \PY{n}{failed\PYZus{}count}\PY{p}{,} \PY{n}{trimmed\PYZus{}counts}\PY{p}{,} \PY{n}{b}\PY{p}{)}
\PY{n}{x} \PY{o}{=} \PY{n}{get\PYZus{}x\PYZus{}distribution\PYZus{}actual}\PY{p}{(}\PY{n}{A}\PY{p}{,} \PY{n}{b}\PY{p}{)}
\PY{n+nb}{print}\PY{p}{(}\PY{l+s+sa}{f}\PY{l+s+s2}{\PYZdq{}}\PY{l+s+s2}{The HHL circuit failed }\PY{l+s+si}{\PYZob{}}\PY{n}{failed\PYZus{}count}\PY{o}{/}\PY{n}{shots}\PY{o}{*}\PY{l+m+mi}{100}\PY{l+s+si}{\PYZcb{}}\PY{l+s+s2}{\PYZpc{} of the time.}\PY{l+s+s2}{\PYZdq{}}\PY{p}{)}
\PY{n+nb}{print}\PY{p}{(}\PY{l+s+s1}{\PYZsq{}}\PY{l+s+s1}{|x\PYZgt{} prob. HHL   :}\PY{l+s+s1}{\PYZsq{}}\PY{p}{,} \PY{n}{np}\PY{o}{.}\PY{n}{array2string}\PY{p}{(}\PY{n}{x\PYZus{}HHL}\PY{p}{,} \PY{n}{formatter}\PY{o}{=}\PY{p}{\PYZob{}}\PY{l+s+s1}{\PYZsq{}}\PY{l+s+s1}{float\PYZus{}kind}\PY{l+s+s1}{\PYZsq{}}\PY{p}{:} \PY{k}{lambda} \PY{n}{x}\PY{p}{:} \PY{l+s+sa}{f}\PY{l+s+s2}{\PYZdq{}}\PY{l+s+si}{\PYZob{}}\PY{n}{x}\PY{l+s+si}{:}\PY{l+s+s2}{.3f}\PY{l+s+si}{\PYZcb{}}\PY{l+s+s2}{\PYZdq{}}\PY{p}{\PYZcb{}}\PY{p}{)}\PY{p}{)}
\PY{n+nb}{print}\PY{p}{(}\PY{l+s+s1}{\PYZsq{}}\PY{l+s+s1}{|x\PYZgt{} prob. actual:}\PY{l+s+s1}{\PYZsq{}}\PY{p}{,} \PY{n}{np}\PY{o}{.}\PY{n}{array2string}\PY{p}{(}\PY{n}{np}\PY{o}{.}\PY{n}{real}\PY{p}{(}\PY{n}{x}\PY{p}{)}\PY{p}{,} \PY{n}{formatter}\PY{o}{=}\PY{p}{\PYZob{}}\PY{l+s+s1}{\PYZsq{}}\PY{l+s+s1}{float\PYZus{}kind}\PY{l+s+s1}{\PYZsq{}}\PY{p}{:} \PY{k}{lambda} \PY{n}{x}\PY{p}{:} \PY{l+s+sa}{f}\PY{l+s+s2}{\PYZdq{}}\PY{l+s+si}{\PYZob{}}\PY{n}{x}\PY{l+s+si}{:}\PY{l+s+s2}{.3f}\PY{l+s+si}{\PYZcb{}}\PY{l+s+s2}{\PYZdq{}}\PY{p}{\PYZcb{}}\PY{p}{)}\PY{p}{)}
\PY{n+nb}{print}\PY{p}{(}\PY{l+s+s1}{\PYZsq{}}\PY{l+s+s1}{Normalized Error:}\PY{l+s+s1}{\PYZsq{}}\PY{p}{,} \PY{n}{np}\PY{o}{.}\PY{n}{linalg}\PY{o}{.}\PY{n}{norm}\PY{p}{(}\PY{n}{x\PYZus{}HHL} \PY{o}{\PYZhy{}} \PY{n}{x}\PY{p}{)}\PY{p}{)}
\end{Verbatim}
\end{tcolorbox}

    \begin{Verbatim}[commandchars=\\\{\}]
Eigenvalues of A': [0.138 0.225 0.267 1.000]
Condition number of A': 7.2299263194302705
Rotation Angles: [0.70 2.80 1.80 2.89]
The HHL circuit failed 59.1796875\% of the time.
|x> prob. HHL   : [0.880 0.100 0.019]
|x> prob. actual: [0.679 0.302 0.019]
Normalized Error: 0.28464306925482824
    \end{Verbatim}

    Thus, rather naively, with simple untuned hyperparameters and a
relatively large error tolerance, performing the psuedoinverse, HHL
algorithm, and statistical sampling still allows us to learn quite alot
about the optimal fits \(\ket{\lambda}\). Upon comparing the probability
distributions of our quantum calculatted fitting and the classical
answer, we can see that this method offers a valuable insight into
fitting our least squares problem.

    \begin{tcolorbox}[breakable, size=fbox, boxrule=1pt, pad at break*=1mm,colback=cellbackground, colframe=cellborder]
\prompt{In}{incolor}{ }{\boxspacing}
\begin{Verbatim}[commandchars=\\\{\}]

\end{Verbatim}
\end{tcolorbox}


    % Add a bibliography block to the postdoc
    
    
    
\end{document}
